\section{Note del corso}
Il corso di Reti di Elaboratori segue i capitoli del Kurose-Ross.
I tre macro-argomenti sono:
\begin{enumerate}
  \item Fondamenti di fisica
  \item TCP/IP
  \item Reti wireless, sicurezza delle reti, e così via \ldots
\end{enumerate}

\section{Componenti della rete Internet}
Centinaia di milioni di dispositivi (\textbf{hosts} o endsystems) che ospitano \textbf{network apps} sono interconnessi insieme ai router tramite \textit{connection links}. 

\begin{itemize}
    \item
        \textit{Hosts} (o \textit{End-Systems}), sui quali girano applicazioni web. \textit{End Systems} perché sono le foglie di un albero di collegamenti che passano per diversi \textit{router} (i quali permettono lo scambio di informazioni tra i vari hosts);
    \item
        \textit{Communication Links} (fibra, radio, satellite, e così via \ldots), a cui è associato un determinato insieme di frequenze, detto \textit{banda}. Conoscendo la banda è possibile ricavare il quantitativo di bit massimo che posso far attraversare, il \textit{tasso di trasmissione}. La \textit{collisione} è la sovrapposizione di pacchetti di comunicazione da sorgenti diverse, e causa la mancata decodifica dei pacchetti da parte del router;
    \item
        \textit{router} è l'elemento cui viene affidato il compito di gestire l'inoltro dei pacchetti dalla sorgente alla destinazione. È formato da link di ingresso, sui quali esegue il \textit{forwarding} verso i link d'uscita. Per poterlo fare, necessita di un'informazione, l'indirizzo del destinatario. Controlla quindi una \textit{tabella d'inoltro}, composta da due colonne: l'indirizzo del destinatario e \textit{next hop}. Seguendo ogniqualvolta necessario i riferimenti associati all'indirizzo richiesto sulla tabella, il pacchetto arriva a destinazione;
    \item
        \textit{Protocolli} implementano funzionalità necessarie per la rete; controllano la ricezione e la trasmissione di messaggi. Alcuni esempi di protocolli sono TCP, IP, HTTP, ETHERNET e SKYPE.
    \item
        \textit{Standard}: definiscono le regole attraverso cui vengono applicati tutti i protocolli in \textit{internet}.
        \begin{itemize}
            \item L'\textit{RFC} (\textit{request for comments}) contiene tutti gli standard pubblicamente riconosciuti e in vigore su internet.
            \item Lo \textit{IETF} (\textit{internet engineering task force}) viene consultato ogniqualvolta giunge una richiesta di inclusione di un nuovo standard. Dopo essere stato sufficientemente vagliato, lo standard può essere incluso tra gli RFC.
        \end{itemize}
\end{itemize}

\section{Rete Logica e Rete Fisica}
\begin{itemize}
    \item Una \textbf{Rete Fisica} è il diagramma che costituisce l'assetto fisico di una rete. Contiene una serie di nodi - ciascun elemento di rete - legati tra loro da un mezzo trasmissivo;
    \item Una \textbf{Rete Logica} è il diagramma che costituisce l'assetto logico di una rete. Partendo da quello fisico, in questo caso, per ogni nodo possono essere applicate delle policies che permettono - o meno - la trasmissione di dati attraverso mezzi trasmissivi, verso un dato elemento di rete ad esso fisicamente collegato.
\end{itemize}

\section{Internet: Servizi di comunicazione}
I protocolli definiscono come avvengono i colloqui tra dispositivi attraverso specifici set di regole.

\subsection{TCP}
Il \textit{Transmission Control Protocol} è un protocollo di rete che si occupa di controllo di rendere affidabile la comunicazione dati in rete tra mittente e destinatario.
In particolare, questo protocollo fornisce:
    \begin{itemize}
        \item trasferimento dati ordinato e affidabile
        \begin{itemize}
            \item permette riconoscimento e ritrasmissione di eventuali perdite di dati
        \end{itemize}
        \item controllo del flusso
        \begin{itemize}
            \item il mittente non sovraccaricherà il ricevente
        \end{itemize}
        \item controllo della congestione
        \begin{itemize}
            \item il mittente rallenterà la velocità di invio quando la rete è congestionata
        \end{itemize}
    \end{itemize}

    \subsubsection{Handshake}
    Si definisce \textit{handshake}(lett. \textit{stretta di mano}) il processo attraverso il quale due calcolatori stabiliscono le regole comuni, ovvero la velocità, i protocolli di compressione, di criptazione, di controllo degli errori etc.
    
    Prima di iniziare la connessione vera e propria tra due macchine si crea questo tipo di connessione che consiste nella trasmissione dei pacchetti per regolare i parametri di connessione.

\subsection{UDP} (\textit{User Datagram Protocol}
A differenza del protocollo TCP non esegue alcun tipo di \textit{handshake} o controllo: questoè utile (o necessario) per alcuni tipi di applicativi che fanno loro componente fondamentale la rapidità d trasmissione di dati, come i servizi di streaming o voip, per i quali diventa di fondamentale importanza la rapidità di ricezione dei pacchetti, piuttosto che la loro qualità.

\section{Trasmissione Dati}
\begin{itemize}
    \item \textbf{Commutazione di Circuito}: si stabilisce un circuito (percorso) tra mittente e destinatario, soltanto dopo aver appurato che le risorse necessarie per effettuare l'effettiva trasmissione dei dati siano disponibili.
    Qualora ci fossero diversi utenti, la suddivisione delle risorse potrebbe avvenire attraverso due modalità:
    \begin{itemize}
        \item \textbf{FDM}: \textit{Frequence Division Multiplexing}, che consiste nel suddividere le risorse in canali, uno per utente. In questo modo ogni utente avrà 1/x risorse a disposizione, dove x è il numero di utenti che utilizza la rete;
        \item \textbf{TDM}: \textit{Time Division Multiplexing}, che consiste nel suddividere le risorse in base al tempo. Ogni utente utilizza la totalità delle risorse disponibili per un determinato lasso di tempo, a conclusione del quale l'utilizzo delle risorse passa all'utente successivo fino a ricominciare il ciclo con il primo utente.
    \end{itemize}
    \textbf{ESEMPIO}. Vogliamo ricavarci il tempo necessario per trasmettere 640.000 bit da un host A ad un host B, sapendo che la rete ha una banda di 1,536Mbps.
    \begin{equation}
        1,536/24 = 64K \to 64.000/640.000 = 1/10s
    \end{equation}
    \textbf{ESEMPIO}. Trasmettiamo 7,5Mbit con una banda massima di 1,5Mbps.
    \begin{equation}
        3*7,5/1,5 = 3*5 = 15s
    \end{equation}
    
    \item \textbf{Commutazione di Pacchetto}: ogni informazione da trasmettere viene suddivisa in \textit{n} pacchetti. Ogni host condivide in questo caso le stesse risorse di rete, ogni pacchetto utilizza tutta la banda disponibile e le risorse vengono adoperate soltanto quando necessario.
    Solitamente le code di pacchetti sono gestite in ordine di arrivo.
    \textbf{ESEMPIO}. Trasmettiamo 7,5Mbit con una banda massima di 1,5Mbps.
    \begin{enumerate}
        \item Dividiamo l'informazione da trasmettere in 5000 pacchetti da 1500 bit;
        \item Occorre quindi 1ms per trasmettere pacchetti su un link
        \item Ogni link lavora in parallelo (\textit{pipelining})
        \item Il delay è ridotto da 15s (nel caso dello stesso esempio applicato in un paradigma a commutazione di circuito) a 5,003s.
    \end{enumerate}
\end{itemize}
Il paradigma a commutazione di pacchetto permette a più utenti di utilizzare - in tutta la sua banda e potenza - la stessa rete.

\section{Tipi di Sorgenti}
Le sorgenti di trasmissione dei dati possono utilizzare tassi di frequenza di trasmissione diversi:
\begin{itemize}
    \item \textbf{Tasso costante}: nel caso della telefonia, per esempio, i pacchetti hanno una dimensione fissa e vengono trasmessi ad intervalli regolari. La frequenza è di 64Kbps;
    \item \textbf{Tasso variabile}: facendo riferimento all'esempio precedente, alcuni software gestiscono in maniera ottimale la trasmissione dei dati: associaziono ad una variabile il rumore di sottofondo. Ogniqualvolta riconosceranno - utilizzando quella variabile - che l'utente non sta parlando, non invieranno alcun pacchetto.
\end{itemize}



\section{access networks and physical media	}
..Questi sistemi sono interconnessi a Internet tramite \textit{reti di accesso}. Possono essere:
\begin{itemize}
	\item residential access nets
	\item institutional access networks (school, company)
	\item mobile access networks
\end{itemize}
Elementi essenziali della rete di accesso sono la banda (bit/s), l'affidabilità (bit error rates) e se è dedicata o condivisa.
\paragraph{Punti di accesso}
Un altro elemento fondamentale è il punto di accesso alla rete: essa dev'essere in grado di riconoscere che si sta degradando il segnale e individuare un altro punto di accesso al quale ci stiamo avvicinando [--> non degradare la qualità della rete quando si è in movimento].
\\

\section{Transmission across a physical link}
Ciò che viene trasmesso è una sequenza di bit che codifica il contenuto.
\begin{itemize}
	\item \textbf{Bits}: propagate between transmitter and receiver
	\item \textbf{Physical link}: what lies between transmitter \& receiver
	\item \textbf{guided media}: signals propagate in solid media [...]
	\item :missing item:
\end{itemize}

Il segnale inviato subisce un'attenuazione durante la propagazione dell'informazione.
Ciò non avviene in maniera uniforme; bisogna anche mettere in conto eventuali interferenze elettromagnetiche (spesso introdotte dalle stesse apparecchiature utilizzate per la connessione). Il risultato è che la sequenza ricevuta potrebbe non essere corretta.\\
\textit{Bit error rate}: Varia da un dispositivo all'altro.
In questo caso, occorre che il dispositivo sia error-aware, individuare la presenza di errori e tentare di correggerli (possibile fino a un certo numero di bit, possibile sia necessario scartare il pacchetto).

\subsection{Codifica NRZ}
\textit{NRZ: Non Return to Zero}. Il tempo è diviso in slot - unità temporali che corrispondono ad un bit.
Ogni bit ha associato un valore stabile per la sua intera durata (1: High, 0: Low).
Problema: sincronizzazione con il ricevitore (nessuna transizione nel caso di sequenze di 0 o 1).
\subsection{Codifica Manchester}
Utilizzata in Ethernet.
È forzata una transizione basso-alto (se 0) o alto-basso (se 1) in corrispondenza di ogni bit.

\section{Tecnologie di accesso odierne}
\subsection{Dial-up modem}
Velocità trasmissiva: ~56Kbps
Utilizza la stessa infrastruttura telefonica: non \textit{always-on}, impossibile chiamare+navigare.
La rete telefonica arrivava al central office, veniva convertita e raggiungeva direttamente l'InternetServiceProvider.

\subsection{Digital Subscriber Line}
Anch'essa utilizza un'infrastruttura telefonica esistente.\\Spesso ADSL (A: asimmetrico).\\
Essendo la frequenza del segnale vocale sempre in un intervallo tra 0 e 4KHz, il segnale può essere campionato fino a 8M volte/s ed essere successivamente ricostruito; in quella frequenza il segnale è trasmesso \textit{solo} attraverso via telefonica.\\
Le risorse del canale sono divise attraverso \textit{Frequency Division Multiplexing}.\\
Velocità : fino a 1Mbps up (tipicamente < 256K), fino a 8Mbps (tipicamente < 1M) down. La grande differenza di velocità trasmissiva è causata da un paradigma client/server che dà la precedenza al download dei dati piuttosto che alla loro trasmissione.
\\
Tecnologie di tipo DSL sono progettate per raggiungere distanze non molto elevate (fino a 4/5 km).
\\
\paragraph{ADSL loop extender} 
Dispositivo posizionato tra il cliente e il central office dalla compagnia telefonica per estendere la distanza ed aumentare la velocità di trasmissione.
\\
\subsection{Residential access: cable modems}
Utilizzato spesso negli USA.
Non utilizza l'infrastruttura telefonica ma quella della TV via cavo.
\textbf{HTC}:\textit{hybrid fiber coax}, fino a 20Mbps down, 2Mbps up
	\subsection{Fibra ottica}
	Collegamenti ottici dal central office a casa.
	Due tipologie di accesso:
	\begin{itemize}
		\item Passive Optical Network - un'unica fibra \& uno splitter ottico che smista le informazioni.\\
		Efficienti: fino a 32 utenti per un solo cavo, bassi costi
		\\Più utenze ricevono le trasmissioni per un determinato utente --> informazioni sono crittate per permettere solo all'utente interessato possa leggerle.\\%[...]
		
		\item Active Optical Network
	\end{itemize}
	\subsection{Ethernet}
	Evoluzione che risale agli '80; tecnologia basata su formati e protocolli che sono rimasti immutati.
	Elemento centrale (switch) + diverse utenze di vario tipo: endsystem di entrambi i tipi + router che connette all'ISP.
	\subsection{Wireless Access Networks}
	In questo caso, le informazioni sono trasmesse andando a modulare opportunamente %COSA?
	Mezzo radio condiviso trasmette lungo un determinato raggio di azione ai dispositivi.
	\begin{itemize}
		\item dati non possono essere "in chiaro".
		\item rischio elevato di accavallamento di trasmissioni
	\end{itemize}
	Protocolli appositi individuano e "scavallano" collisioni.
	
	
	\section{Internet Structure: network of networks}
	La maggior parte degli ISP hanno copertura regionale/nazionale; richiedono accesso ai pochissimi service provider \textit{Tier One}, che hanno copertura mondiale (Verizon, Sprint, AT\&T,). Questi ISP non popolano letteralmente tutte le zone del mondo; interconnettono i router tramite POP (point of presence).\\ 
	
	Gli ISP Tier 2 fanno degl accordi con i t1 (sono dei clienti!) e a loro volta spesso offrono copertura maggiore a ISP locali (tier3).