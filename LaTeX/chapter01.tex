\section{Note del corso}
Il corso di Reti di Elaboratori segue i capitoli del Kurose-Ross.
I tre macro-argomenti sono:
\begin{enumerate}
  \item Fondamenti di fisica
  \item TCP/IP
  \item Reti wireless, sicurezza delle reti, e così via \ldots
\end{enumerate}

\section{Componenti della rete Internet}
Centinaia di milioni di dispositivi (\textbf{hosts} o endsystems) che ospitano \textbf{network apps} sono interconnessi insieme ai router tramite \textit{connection links}. 

\begin{itemize}
    \item
        \textit{Hosts} (o \textit{End-Systems}), sui quali girano applicazioni web. \textit{End Systems} perché sono le foglie di un albero di collegamenti che passano per diversi \textit{router} (i quali permettono lo scambio di informazioni tra i vari hosts);
    \item
        \textit{Communication Links} (fibra, radio, satellite, e così via \ldots), a cui è associato un determinato insieme di frequenze, detto \textit{banda}. Conoscendo la banda è possibile ricavare il quantitativo di bit massimo che posso far attraversare, il \textit{datarate} (o \textit{tasso di trasmissione}). La \textit{collisione} è la sovrapposizione di pacchetti di comunicazione da sorgenti diverse, e causa la mancata decodifica dei pacchetti da parte del router;
    \item
        \textit{router} è l'elemento cui viene affidato il compito di gestire l'inoltro dei pacchetti dalla sorgente alla destinazione. È formato da link di ingresso, sui quali esegue il \textit{forwarding} verso i link d'uscita. Per poterlo fare, necessita di un'informazione, l'indirizzo del destinatario. Controlla quindi una \textit{tabella d'inoltro}, composta da due colonne: l'indirizzo del destinatario e \textit{next hop}. Seguendo ogniqualvolta necessario i riferimenti associati all'indirizzo richiesto sulla tabella, il pacchetto arriva a destinazione;
    \item
        \textit{Protocolli} implementano funzionalità necessarie per la rete; controllano la ricezione e la trasmissione di messaggi. Alcuni esempi di protocolli sono TCP, IP, HTTP, ETHERNET e SKYPE.
    \item
        \textit{Standard}: definiscono le regole attraverso cui vengono applicati tutti i protocolli in \textit{internet}.
        \begin{itemize}
            \item L'\textit{RFC} (\textit{request for comments}) contiene tutti gli standard pubblicamente riconosciuti e in vigore su internet.
            \item Lo \textit{IETF} (\textit{internet engineering task force}) viene consultato ogniqualvolta giunge una richiesta di inclusione di un nuovo standard. Dopo essere stato sufficientemente vagliato, lo standard può essere incluso tra gli RFC.
        \end{itemize}
\end{itemize}

\section{Rete Logica e Rete Fisica}
\begin{itemize}
    \item Una \textbf{Rete Fisica} è il diagramma che costituisce l'assetto fisico di una rete. Contiene una serie di nodi - ciascun elemento di rete - legati tra loro da un mezzo trasmissivo;
    \item Una \textbf{Rete Logica} è il diagramma che costituisce l'assetto logico di una rete. Partendo da quello fisico, in questo caso, per ogni nodo possono essere applicate delle policies che permettono - o meno - la trasmissione di dati attraverso mezzi trasmissivi, verso un dato elemento di rete ad esso fisicamente collegato.
\end{itemize}

\section{Internet: Servizi di comunicazione}
I protocolli definiscono come avvengono i colloqui tra dispositivi attraverso specifici set di regole.

\subsection{TCP}
Il \textit{Transmission Control Protocol} è un protocollo di rete che si occupa di rendere affidabile la trasmissione dati in rete tra mittente e destinatario.
In particolare, questo protocollo fornisce:
    \begin{itemize}
        \item trasferimento dati ordinato e affidabile: permette riconoscimento e ritrasmissione di eventuali perdite di dati
        \item controllo del flusso: il mittente non sovraccaricherà il ricevente
        \item controllo della congestione: il mittente rallenterà la velocità di invio quando la rete è congestionata
    \end{itemize}

    \subsubsection{Handshake}
    Si definisce \textit{handshake} (lett. \textit{stretta di mano}) il processo attraverso il quale due calcolatori stabiliscono le regole comuni, ovvero la velocità, i protocolli di compressione, di criptazione, di controllo degli errori etc.
    
    Prima di iniziare la connessione vera e propria tra due macchine si crea questo tipo di connessione che consiste nella trasmissione dei pacchetti per regolare i parametri di connessione.

\subsection{UDP}
Lo \textit{User Datagram Protocol}, a differenza del protocollo TCP non esegue alcun tipo di \textit{handshake} o controllo: questo è utile (o necessario) per alcuni tipi di applicativi che fanno loro componente fondamentale la rapidità di trasmissione di dati, come i servizi di streaming o voip, per i quali diventa di fondamentale importanza la rapidità di ricezione dei pacchetti, piuttosto che la loro qualità.

\section{Trasmissione Dati}
\begin{itemize}
    \item \textbf{Commutazione di Circuito}: si stabilisce un circuito (percorso) tra mittente e destinatario, soltanto dopo aver appurato che le risorse necessarie per effettuare l'effettiva trasmissione dei dati siano disponibili.
    Qualora ci fossero diversi utenti, la suddivisione delle risorse potrebbe avvenire attraverso due modalità:
    \begin{itemize}
        \item \textbf{FDM}: \textit{Frequence Division Multiplexing}, che consiste nel suddividere le risorse in canali, uno per utente. In questo modo ogni utente avrà 1/x delle risorse totali a propria disposizione, dove x è il numero di utenti che utilizza la rete;
        \item \textbf{TDM}: \textit{Time Division Multiplexing}, che consiste nel suddividere le risorse in base al tempo. Ogni utente utilizza la totalità delle risorse disponibili per un determinato lasso di tempo, a conclusione del quale l'utilizzo delle risorse passa all'utente successivo fino a ricominciare il ciclo con il primo utente.
    \end{itemize}
    \textbf{ESEMPIO}. Vogliamo ricavarci il tempo necessario per trasmettere 640.000 bit da un host A ad un host B, sapendo che la rete ha una banda di 1,536Mbps.
    \begin{equation}
        1,536/24 = 64K \to 64.000/640.000 = 1/10s
    \end{equation}
    \textbf{ESEMPIO}. Trasmettiamo 7,5Mbit con una banda massima di 1,5Mbps.
    \begin{equation}
        3*7,5/1,5 = 3*5 = 15s
    \end{equation}
    \newpage
    
    \item \textbf{Commutazione di Pacchetto}: ogni informazione da trasmettere viene suddivisa in \textit{n} pacchetti. Ogni host condivide in questo caso le stesse risorse di rete, ogni pacchetto utilizza tutta la banda disponibile e le risorse vengono adoperate soltanto quando necessario.
    Solitamente le code di pacchetti sono gestite in ordine di arrivo. \\ \\
    \textbf{ESEMPIO}. Trasmettiamo 7,5Mbit con una banda massima di 1,5Mbps.
    \begin{enumerate}
        \item Dividiamo l'informazione da trasmettere in 5000 pacchetti da 1500 bit;
        \item Occorre quindi 1ms per trasmettere pacchetti su un link
        \item Ogni link lavora in parallelo (\textit{pipelining})
        \item Il delay è ridotto da 15s (nel caso dello stesso esempio applicato in un paradigma a commutazione di circuito) a 5,003s.
    \end{enumerate}
\end{itemize}
Il paradigma a commutazione di pacchetto permette a più utenti di utilizzare - in tutta la sua banda e potenza - la stessa rete.

\section{Tipi di Sorgenti}
Le sorgenti di trasmissione dei dati possono utilizzare tassi di frequenza di trasmissione diversi:
\begin{itemize}
    \item \textbf{Tasso costante}: nel caso della telefonia, per esempio, i pacchetti hanno una dimensione fissa e vengono trasmessi ad intervalli regolari. La frequenza è di 64Kbps;
    \item \textbf{Tasso variabile}: facendo riferimento all'esempio precedente, alcuni software gestiscono in maniera ottimale la trasmissione dei dati: associaziono ad una variabile il rumore di sottofondo. Ogniqualvolta riconosceranno - utilizzando quella variabile - che l'utente non sta parlando, non invieranno alcun pacchetto.
\end{itemize}

\section{Network Edge}
Come già anticipato, ogni \textit{end-system} (\textit{host}) esegue applicativi. Questo può avvenire secondo diverse modalità:
\begin{itemize}
	\item \textbf{Client-Server}: un host (\textit{server}) esegue applicativi adibiti alla manipolazione di informazioni, ma soprattutto allo scambio di informazioni con altri host (\textit{client}). Quindi, essenzialmente, i \textit{client} effettuano richieste ai \textit{server}, che eseguono servizi perennemente in esecuzione (\textit{always-on});
	\item \textbf{Peer-Peer}: non esiste differenziazione tra \textit{server} e \textit{client}, perché in questo caso gli host eseguono le stesse attività, ma comunicando tra loro (come nei casi di servizi come \textit{Torrent}, \textit{Skype}, e così via \ldots).
\end{itemize}

\section{Accesso alla rete}
Ogni dispositivo può essere connesso ad internet tramite diverse \textit{reti di accesso}. Possono essere:
\begin{itemize}
	\item Reti residenziali
	\item Reti istituzionali (scuole, aziende, e così via \ldots)
	\item Reti mobili (come quelle utilizzate dagli smartphones)
\end{itemize}
Elementi essenziali della rete di accesso sono la banda (bit/s), l'affidabilità (bit error rates) e se è dedicata o condivisa.
\paragraph{Punti di accesso mobili}
Un altro elemento fondamentale è il punto di accesso alla rete: essa dev'essere in grado di riconoscere che si sta degradando il segnale e individuare un altro punto di accesso al quale ci stiamo avvicinando, così da evitare un'eventuale degradazione della qualità della rete quando si è in movimento.

\section{Trasmissione attraverso link fisici}
Ciò che viene trasmesso è una sequenza di bit che codifica il contenuto.
\begin{itemize}
	\item \textbf{Bit}: valori che si propagano tra il trasmettitore e il ricevitore
	\item \textbf{Link}: tracce attraverso cui passano i bit
	\begin{itemize}
		\item \textbf{Tracce guidate}: cavi fisici (fibra, e così via \ldots)
		\item \textbf{Tracce radio}: le informazioni vengono trasmesse attraverso onde radio
	\end{itemize}
\end{itemize}

Le informazioni trasmesse tramite onde radio subisce un'\textit{attenuazione} durante la propagazione dell'informazione, e quindi una \textit{distorsione}.
Ciò non avviene in maniera uniforme; bisogna anche mettere in conto eventuali interferenze elettromagnetiche (spesso introdotte dalle stesse apparecchiature utilizzate per la connessione). Il risultato è che la sequenza ricevuta potrebbe non essere corretta. Questo introduce la definizione di \textit{bit error rate}, ovvero la frequenza con cui si verifica un errore nella trasmissione delle informazioni. Minore è il suo valore, migliore e più stabile è la rete.
In questo caso, occorre che il dispositivo sia \textit{error-aware}, così da essere autonomo nell'individuare la presenza di errori e tentare di correggerli (possibile fino a un certo numero di bit, altrimenti viene scartato il pacchetto e richiesto indietro).

\subsection{Codifica NRZ}
\textit{NRZ: Non Return to Zero}. Il tempo è diviso in slot - unità temporali che corrispondono ad un bit.
Ogni bit ha associato un valore stabile per la sua intera durata (1: High, 0: Low).
Il problema in questo tipo di codifica è la spesso asicronizzazione tra trasmettitore e ricevitre.

\subsection{Codifica Manchester}
Utilizzata in Ethernet. Essenzialmente viene associato un valore ad ogni transizione. Quando la frequenza di trasmissione va dal basso all'alto - di intensità - il valore associato è 1, dall'alto al basso, invece, 0.

\section{Tecnologie di accesso}
\subsection{Dial-up modem}
Velocità trasmissiva: 56Kbps (ca.). Utilizza la stessa infrastruttura telefonica: non \textit{always-on}, impossibile poter usufruire della linea per internet e telefonia allo stesso tempo.
Il modem reindirizza le richieste al \textit{central office}, che - qualora si cerchi di navigare in internet - converte la richiesta e la reindirizza all'\textit{ISP} (\textit{Internet Service Provider}).

\subsection{Digital Subscriber Line}
Anch'essa utilizza un'infrastruttura telefonica esistente. Spesso chiamata anche A-DSL (la \textit{A} indica l'\textit{asimmetria} tra la velocità di trasmissione modem-\textit{central office} e \textit{central office}-modem).
Essendo la frequenza del segnale vocale sempre in un intervallo tra 0 e 4KHz, il segnale può essere campionato fino a 8M volt/s ed essere successivamente ricostruito; in quella frequenza il segnale è trasmesso \textit{solo} attraverso via telefonica. Tutte le trasmissioni fuori da quell'intervallo vengono quindi campionate come richieste per internet.
Le risorse del canale sono divise attraverso \textit{FDM} (\textit{Frequency Division Multiplexing}).\\
Velocità: fino a 1Mbps up (tipicamente $<$ 256Kbps) e a 8Mbps down (tipicamente $<$ 1Mbps) down. La grande differenza di velocità trasmissiva è causata da un paradigma client/server che dà la precedenza al download dei dati piuttosto che alla loro trasmissione.
\\
Tecnologie di tipo DSL sono progettate per raggiungere distanze non molto elevate (fino a 4/5 km).
\paragraph{ADSL loop extender} 
Dispositivo posizionato tra il cliente e il central office dalla compagnia telefonica per estendere la distanza ed aumentare la velocità di trasmissione.

\subsection{HFC (in \textit{USA})}
Velocità: fino a 20Mbps down, 2Mbps up.
Utilizza - invece dell'infrastruttura telefonica - quella della TV via cavo.
Sostanzialmente il \textit{central office}, in questo caso il \textit{cable headend}, gestisce degli anelli di fibra, composti da moltissimi nodi che li collegano ad ogni abitazione. A questo punto, la connessione passa per uno splitter (che utilizza l'\textit{FDM}), che la indirizza alla TV o al dispositivo che richiede la connessione internet. 

\subsection{Fibra ottica}
Collegamenti ottici dal \textit{central office} all'abitazione. Raggiunge distanze lunghissime praticamente senza alcuna distorsione del segnale.
Due tipologie di accesso:
\begin{itemize}
	\item \textit{Passive Optical Network} (\textit{PON}): un'unica fibra e uno splitter ottico che smista le informazioni verso diverse abitazioni. Comporta costi decisamente bassi, a scompenso però della qualità:
	\begin{itemize}
		\item Si possono spesso verificare \textit{collisioni}, evitate attraverso il \textit{TDM}
		\item Essendo una linea condivisa si presenta la necessità di dover criptare i messaggi
		\item Non raggiunge distanze massime
		\item Diventa più difficile isolare eventuali problemi
		\item Velocità di accesso non ottimale
	\end{itemize}
	\item \textit{Active Optical Network} (\textit{PAN})
\end{itemize}

\subsection{Ethernet}
Evoluzione che risale agli anni '80; tecnologia basata su formati e protocolli che sono rimasti immutati, se non per supportare il decisivo incremento della velocità di navigazione.
Uno switch - con ingressi ethernet, appunto - viene frapposto tra router/modem e diversi host.

\subsection{Wireless Access Networks}
In questo caso, le informazioni sono modulate opportunamente, per raggiungere le frequenze delle onde radio supportate dal ricetrasmettitore radio.
Ne esistono due tipi:
\begin{enumerate}
	\item \textit{Wireless LAN} (o \textit{WiFi}) 802.11b/g: raggiunge una velocità comresa tra gli 11Mbps e i 54Mbps.
	\item \textit{Wide-area Wireless Access}: mezzo radio adibito condiviso che raggiunge una velocità compresa tra 1Mbps (\textit{EVDO}/\textit{HSDPA}) e diverse decine di Mbps (\textit{LTS}).
\end{enumerate}
Essendo la rete condivisa, neccessità di essere criptata. Non utilizza né \textit{FDM} né \textit{TDP}, ma un sistema di gestione proprietario e specifico.

\section{Reti Mobili}
\begin{center}
	\begin{tabular}{| c | c | c | c | c |}
		\hline
		\textbf{Generazione} & \textbf{Simbolgia} & \textbf{Tecnologia } & \textbf{DL Massima} in Mbit/s & \textbf{DL Tipica} in Mbit/s \\ \hline
		2G & G & GPRS & 0,1 & 0,03 \\ \hline
		2G & E & EDGE & 0,3 & 0,1 \\ \hline
		3G & 3G & 3G (basic) & 0,3 & 0,1 \\ \hline
		3G & H & HSPA & 7,2 & 1,5 \\ \hline
		3G & H+ & HSPA+ & 21 & 1 \\ \hline
		3G & H+ & DC-HSPA+ & 42 & 8 \\ \hline
		4G & 4G & LTE & 100 & 15 \\ \hline
	\end{tabular}
\end{center}

\section{Componenti attuali}
\begin{enumerate}
	\item \textit{DSL Fast-Technology} (combinata con fibra): 1Gbps
	\item \textit{Ethernet}: 25-40Gbps (con più linee 100-400Gbps)
	\item \textit{WiFi IEEE 802.11ac}: 1Gbps
	\item \textit{Fibra}: diverse decine di Tbps
	\item \textit{Reti mobili}: diverse centinaia di Mbps
\end{enumerate}

\section{Internet Structure: network of networks}
Internet ha una struttura gerarchica, composta da moltissimi \textit{ISP} (\textit{Internet Service Provider}).
La prima categoria è composta dai \textit{Tier-1} (Verizon, Sprint, AT\&T, e così via \ldots). Questi sono interconnessi - per maggiore copertura - tramite dei \textit{POP} (\textit{Point of presence}).
Poi ci sono i \textit{Tier-2}, anche loro interconnessi tra loro e con i \textit{Tier-1}.
La gerarchia continua seguendo questo schema, fino a raggiungere gli \textit{ISP} regionali/pronvinciali e, infine, le reti residenziali.
\newpage

\section{Loss \& Delay}
Esistono diversi tipi di ritardi:
\begin{itemize}
	\item \textbf{Processing Delay}: ritardo nel processare i dati ricevuti (per analizzare se contengono errori o meno, per leggere l'header e per instradarli nella coda giusta).
	\item \textbf{Queue Delay}: tempo di attesa prima che il link adibito alla consegna dei dati processi il pacchetto.
	\item \textbf{Transmission Delay}: il tempo necessario a trasmettere tutta la sequenza del pacchetto.
	\begin{itemize}
		\item R = banda link (bps)
		\item L = grandezza pacchetto (bit)
		\item a = frequenza di arrivo pacchetti
		\item \textit{Tempo di trasmissione sul link}: $\frac{L}{R}$
		\item \textit{Intensità di traffico}: $\frac{L\times a}{R}$
	\end{itemize}
	L'arrivo dei pacchetti segue una distribuzione di \textit{Poisson}, e quindi:
	\begin{itemize}
	    \item se l'intensità di traffico è 0 ca., si è verificato un delay piccolo;
	    \item se l'intensità di traffico è 0,5 ca., si è verificato un delay medio;
	    \item se l'intensità di traffico è 1 ca., si è verificato un delay alto e basta pochissimo per congestionare la rete.
	\end{itemize}
	\item \textbf{Propagation Delay}: tempo utilizzato dai bit per propagarsi (alla velocità della luce, ca.) lungo tutta la linea fino alla destinazione.
	\begin{itemize}
		\item d = lunghezza fisica del link
		\item s = tempo di propagazione in media
		\item \textit{Ritardo di propagazione} = d/s
	\end{itemize}
	\item \textbf{Nodal delay} = Processing delay + Queue delay + Transmission delay + Propagation delay
\end{itemize}

Il \textit{trace route program} conta il tempo mandando pacchetti con un contatore \textit{time-to-live} che decrementa di 1 all'incrontro con ogni router.
\\
Il \textit{throughput} indica i bit trasmessi al secondo: il collo di bottiglia sarà il link che avrà il \textit{throughput} minore. Viene quindi inviata la massima quantità pari al \textit{throughput} del link più piccolo.