\section{Protocolli}
\begin{itemize}
    \item \textbf{TCP}: protocollo di trasporto
    \item \textbf{IP}: protocollo di instradamento
\end{itemize}

In una \textit{TCP connection request/response} i protocolli devono specificare:
\begin{itemize}
    \item il formato delle informazioni (ordine dei bit, e quant'altro \ldots)
    \item messaggio da scambiare
    \item ordine degli eventi
\end{itemize}

TCP prevede che prima dello scambio delle informazioni ci sia la \textit{request/response}, un dialogo il cui fine è assicurare che:
\begin{itemize}
    \item i dati siano in ordine e completi
    \item avvenga il \textit{controllo di flusso} (consente di sapere la velocità e la quantità di dati che è possibile trasferire)
    \item avvenga il \textit{controllo di congestione} (i router si accorgono se il flusso in arrivo è maggiore di quello in uscita)
\end{itemize}

A differenza del \textit{TCP}, l'\textit{UDP} non prevede alcuno scambio \textit{request/response}, prima dell'effettivo trasferimento dell'informazione. Il sistema dei \textit{DNS} utilizza \textit{UDP}: quando si ha un \textit{URL}, bisogna ottenerne l'\textit{indirizzo IP} associato, perciò viene richiesta l'informazione tramite il protocollo \textit{UDP}, che garantisce una trasmissione rapida, ma non da certezze che questa vada a buon fine.

\section{Pila TCP/IP}
La \textit{pila TCP/IP} è una pila protocollare, che indica i livelli di protocollo, organizzata a livelli (modularità, così che se variano le regole di un livello, questo cambiamento non tange il funzionamento degli altri).
Tutte le funzionalità logiche si dividono in gruppi, per rendere la progettazione più semplice e più estensibile.
\\
Ciascuna funzionalità ha diversi protocolli.
Il gruppo più in alto corrisponde alla logica applicativa.
Spesso si arricchisce ogni gruppo con microfunzionalità e questo da un servizio aggiuntivo al livello superiore, e queste microfunzionalità possono essere mappate (duplicate) in livelli diversi.
Diverse entità della rete hanno diversi sottoinsiemi delle funzionalità.
Vengono stabiliti gli \textit{access service point} che sono i punti in cui - nell'architettura - c'è il passaggio da un livello ad un altro.
Lo svantaggio di un'architettura di questo tipo è che livelli non adiacenti non hanno la possibilità di comunicare (esiste però un sistema, chiamato \textit{cross layering}, che permette di leggere dati di livelli non adiacenti).
\\
I livelli sono solitamente i seguenti:
\begin{enumerate}
    \item \textbf{Lv "applicativo"} (\textit{software}): molte funzionalità diverse (come la sincronia);
    \item \textbf{Lv di "trasporto"} (\textit{software}): utilizza di base il protocollo \textit{UDP} e controlla che il messaggio venga consegnato al giusto destinatario. Col protocollo \textit{TCP} c'è anche il controllo se il pacchetto è arrivato ed è stato realmente consegnato al giusto destinatario;
    \item \textbf{Lv di "rete"} (\textit{network}): adibito al tentativo di capire il percorso da intraprendere, attraverso algoritmi di instradamenti, per poi inviare il pacchetto attraverso quel percorso;
    \item \textbf{Lv "link"}, o \textbf{datalink} (\textit{hardware}): lavora con i \textit{frame} (\textit{trama}). Verifica se ci sono stati errori di trasmissione, analizzando anche gli indirizzi \textit{MAC} (\textit{Medium Access Control});
    \item \textbf{Lv "fisico"} (\textit{hardware}): enormi eterogeneità.
\end{enumerate}
Sugli \textit{end-systems} ci saranno tutti e 4 i livelli.
Sugli elementi intermedi di commutazione ci saranno solamente i 3 livelli più bassi.
L'\textit{ack} (\textit{acknowledge}) è un tipo di messaggio che viene utilizzato per far capire al sorgente/destinatario che il pacchetto è arrivato.
Ogni livello aggiunge un \textit{header}, che per il livello successivo diventa parte integrante del messaggio stesso da consegnare. Per ciascun livello, ecco l'header:
\begin{itemize}
    \item Applicativo: il messaggio stesso (\textit{messaggio}, o \textit{message});
    \item Trasporto: il \textit{segmento}, o \textit{segment};
    \item Rete: il \textit{pacchetto}, o \textit{datagram};
    \item Datalink: la \textit{trama}, o \textit{frame}.
\end{itemize}
I vantaggi della stratificazione sono la modularità e la gestione dell'eterogeneità.

\section{Storia}
Internet è stato inventato da \textit{Leonard Kleinrock}, quand'era ancora studente ed ebbe l'idea dell'architettura a pacchetto, nel 1961.
Nel 1969, 8 anni dopo aver concepito questa idea, viene attivata la prima rete con 4 nodi (le 4 importanti università della California), utilizzando il protocollo \textit{NCP}.
\\
Nel 1972 viene allargata a 15 nodi.
Negli anni successivi altri studenti hanno avuto l'idea di una rete \textit{wireless}, utilzzando il protocollo \textit{ALOHA}, o anche quella di connettere stampanti - o altri dispositivi -, concependo per la prima volta il concetto di \textit{LAN}.
\\
\textit{Robert E. Kahn} e \textit{Vinton Cerf} capirono l'importanza di mantenere l'eterogeneità della rete. Definirono l'architettura di internet odierna.
\\
Così:
\begin{itemize}
    \item 1979: raggiungimento di 200 nodi;
    \item 1982: nasce l'\textit{SMTP} per le email;
    \item 1983: inventato il \textit{TCP/IP};
    \item 1985: nasce l'\textit{FTP} per lo scambio di file;
    \item 1988: viene aggiunto il controllo di congestione al protocollo TCP.
\end{itemize}
Ciononostante, la rete rimaneva una realtà legata all'ambito accademico.
\\
Nei primi anni '90, \textit{Tim Berners-Lee} ebbe l'idea di web così come oggi è conosciuto, inventando l'\textit{HTTP} e l'\textit{HTML}, sulla base di documenti del 1945 di \textit{Vannevar Bush} che accennava agli \textit{ipertesti}.
\\
Da qui in poi internet ha avuto una crescita esponenziale, sia in velocità che in applicazioni.