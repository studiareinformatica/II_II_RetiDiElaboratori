\section{Informazioni utili}
Esistono diversi tipi di ritardi:
\begin{itemize}
	\item \textbf{Processing Delay}: ritardo nel processare i dati ricevuti (per analizzare se contengono errori o meno e per verificare l'indirizzo del destinatario)
	\item \textbf{Queue Delay}: tempo di attesa prima che il link adibito alla consegna dei dati accolga il pacchetto
	\item \textbf{Transmission Delay}
	\begin{itemize}
		\item R = banda link (bps)
		\item L = grandezza pacchetto (bit)
		\item a = frequenza di arrivo pacchetti
		\item \textit{Tempo di trasmissione sul link} = L/R
		\item \textit{Intensità di traffico} = L*a/R (se pari a 0 non si verifica ritardo nella maniera più assoluta)
	\end{itemize}
	\item \textbf{Propagation Delay}
	\begin{itemize}
		\item d = lunghezza fisica del link
		\item s = tempo di propagazione in media
		\item \textit{Ritardo di propagazione} = d/s
	\end{itemize}
	\item \textbf{Nodal delay} = Processing delay + Queue delay + Transmission delay + Propagation delay
\end{itemize}