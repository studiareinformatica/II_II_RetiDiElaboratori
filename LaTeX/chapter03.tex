\section{Accenni}
Il \textit{Protocollo Applicativo} della pila \textit{TCP/IP} definisce:
\begin{itemize}
    \item i tipi di messaggi scambiati;
    \item la sintassi del messaggio;
    \item la sementica dei campi;
    \item le regole che stabiliscono quando inviare richieste/risposte.
\end{itemize}

\section{HTTP}
Prima di introdurre il concetto di \textit{HTTP} occorre spiegare il paradigma \textit{client-server}:
\begin{itemize}
    \item \textit{Client}: processo client (come un browser, quando richiede una \textit{URL} - \textit{Uniform Resource Locator}) che può fare richieste ed interpretare le risposte;
    \item \textit{Server}: processo server, che fornisce le risposte.
\end{itemize}
Il \textit{client} viene attivato solo quando necessario, mentre invece il \textit{server} è sempre in ascolto.
Un \textit{IP} identifica univocamente un host nella rete (in realtà una scheda di rete).
Attraverso l'indirizzo \textit{IP} si può raggiungere l'host ma non il processo cui occorre effettivamente consegnare il messaggio. Per questo, vengono utilizzate porte differenti. \\
Il binomio \textit{IP-porta} permette a due processi su due host diversi di comunicare. \\
\textit{IANA} è l'\textit{ente} che definisce l'uso delle porte: generalmente, le porte 0-1023 sono riservate, mentre le successive sono per le \textit{user applications}. \\
Sostanzialmente il processo che è in ascolto di messaggi apre un \textit{socket} sulla porta specifica e rimane in attesa di connessioni dall'esterno su quella porta e quel socket (per ogni connessione entrante, nell'header del livello di trasporto viene specificata la porta di destinazione, mentre invece l'IP e l'ID nel livello di rete). \\
Quando viene richiesta una pagina \textit{HTML}, la reazione del protocollo \textit{HTTP} varia in base alla versione:
\begin{itemize}
    \item \textit{HTTP/1.0} faceva tante connessioni quanti erano gli oggetti (caso non persistente) in canali dedicati;
    \item \textit{HTTP/1.1} fa invece una sola connessione che rimane aperta e richiede tutti gli oggetti tramite questa sola connessione.
\end{itemize}
Si definisce \textit{RTT} (\textit{Round Trip Time}), il tempo che un pacchetto impiega ad arrivare al server e tornare indietro.
Per \textit{HTTP} esistono quattro metodi essenziali:
\begin{enumerate}
    \item \textit{GET}: richiede una pagina
    \item \textit{HEAD}: con lo stesso principio del \textit{GET}, riceve soltanto l'header della pagina HTML (utilizzato per motivi di debug)
    \item \textit{POST}: invia dati al server tramite form
    \item \textit{PUT}: carica un file sul server
\end{enumerate}
Esistono poi quattro tipi di header:
\begin{enumerate}
    \item generali
    \item \textit{request} header
    \item \textit{response} header
    \item \textit{entity} header
\end{enumerate}
Un caso specifico da sottolineare è quello dell'\textit{if-modified-since}, che fa parte dei \textit{request} headers: il client specifica una data e richiede al server la risorsa solo se questa è stata modificata dopo quella data. \\
Tra il client e il server ci sono degli intermediari chiamati \textit{proxy-servers}, che mantengono risorse in cache e forniscono le risorse al posto del server interessato. Sostanzialmente, sono delle macchine che lavorano a livello applicativo, con grandi quantità di memoria; leggono le richieste dei client e se possono servire la richiesta lo fanno al posto del server. Chiaramente, sono in costante aggiornamento così da evitare problemi di consistenza. \\
Esistono diversi tipi di \textit{proxy}:
\begin{itemize}
    \item \textit{caching proxy} (più vicino al client);
    \item \textit{forwarding proxy}:
    \begin{itemize}
        \item \textit{trasparent proxy};
        \item \textit{non-trasparent proxy}.
    \end{itemize}
    \item \textit{reverse proxy} (più vicino al server, intercetta le richieste e le invia ai server - in maniera bilanciata -; fanno spesso un lavoro di criptaggio dei messagi).
\end{itemize}
Utilizzando il parametro \textit{no-cache} si specifica il desiderio di non passare per il sistema di \textit{caching} dei \textit{proxy-servers}. \\
Chiaramente tutti i sistemi di \textit{caching} utilizzati seguono una gerarchia, a capo della quale vi è quella locale del proprio browser.

\section{Cookies}
I \textit{cookies} rappresentano una sorta di gettone identificativo, usato dai server web per poter riconoscere i browser durante comunicazioni con il protocollo HTTP usato per la navigazione web.
Tale riconoscimento permette di realizzare meccanismi di autenticazione, di memorizzare dati utili alla sessione di navigazione - come le preferenze sull'aspetto grafico o linguistico del sito -, di associare dati memorizzati dal server - ad esempio il contenuto del carrello di un negozio elettronico -, di tracciare la navigazione dell'utente - ad esempio per fini statistici o pubblicitari. \\
Date le implicazioni per la riservatezza dei naviganti del web, l'uso dei \textit{cookies} è categorizzato e disciplinato negli ordinamenti giuridici di numerosi paesi, tra cui quelli europei, inclusa l'Italia. \\
Nel \textit{cookie} solitamente possiamo trovare quattro attributi:
\begin{itemize}
    \item Nome/valore è una variabile ed un campo obbligatorio;
    \item Scadenza (\textit{expiration date}) è un attributo opzionale che permette di stabilire la data di scadenza del \textit{cookie}. Può essere espressa come data, come numero massimo di giorni oppure come \textit{Now} (adesso) (implica che il \textit{cookie} viene eliminato subito dal computer dell'utente in quanto scade nel momento in cui viene creato) o \textit{Never} (mai) (implica che il \textit{cookie} non è soggetto a scadenza e questi sono denominati persistenti);
    \item Modalità d'accesso (\textit{HttpOnly}) rende il \textit{cookie} invisibile a \textit{javascript} e altri linguaggi \textit{client-side} presenti nella pagina;
    \item Sicuro (\textit{secure}) indica se il \textit{cookie} debba essere trasmesso criptato con \textit{HTTPS}.
\end{itemize}
Il dominio (\textit{domain}) e il percorso (\textit{path}) definiscono l'ambito di visibilità del \textit{cookie}, indicano al browser che il \textit{cookie} può essere inviato al server solo per il dominio e il percorso indicati. Se non specificati, come predefiniti prendono il valore del dominio e del percorso che li ha inizialmente richiesti. \\
\newpage

Un esempio di \textit{cookie} è il seguente:
\begin{lstlisting}
Set-Cookie: LSID=DQAAAK...Eaem_vYg; Domain=docs.foo.com; Path=/accounts; Expires=Wed, 13-Jan-20021 22:23:01 GMT; Secure; HttpOnly
Set-Cookie: HSID=AYQEVn...DKrdst; Domain=.foo.com; Path=/; Expires=Wed, 13-Jan-20021 22:23:01 GMT; HttpOnly
Set-Cookie: SSID=Ap4P...GTEq; Domain=.foo.com; Path=/; Expires=Wed, 13-Jan-20021 22:23:01 GMT; Secure; HttpOnly
\end{lstlisting}

\section{Domain Name System}
Il \textbf{DNS} identifica sia il \textit{database distribuito}, che contiene l'associazione tra nome e IP, sia il protocollo applicativo per l'effettiva risoluzione \textit{nome-IP} (o \textit{IP-nome}). \\
I suoi principali servizi sono:
\begin{itemize}
	\item traduzione tra \textit{hostname} e \textit{IP};
	\item gestione di aliasing degli host;
	\item gestione di aliasing dei mail server;
	\item distribuzione del carico, o \textit{load balancing} (per evitare di sovraccaricare il server, tramite l'associazione di più IP ad un unico nome).
\end{itemize}
È strutturato come database distribuito per le seguenti ragioni:
\begin{itemize}
	\item possibilità di isolamento di eventuali problemi;
	\item gestione del traffico;
	\item gestione delle distanze;
	\item manutenzione.
\end{itemize}
Da qui è opportuno definire il concetto di \textbf{scalabilità}: capacità di un sistema di mantenere un comportamento ottimale alla crescita degli utenti che ne fanno uso. \\
La distribuzione dei DNS avviene in maniera gerarchica:
\begin{enumerate}
    \item \textbf{Root} DNS Servers;
    \item \textbf{TLD} (\textit{Top-level domain}) DNS Servers (ciascuno per gruppi di \textit{TLD} diversi: .com, .org, .edu, e così via \ldots);
    \item \textbf{Authoritative} DNS Servers (per .com vi potrebbero essere amazon.com e yahoo.com, per .edu invece poly.edu, per esempio; e così via \ldots);
    \item \textbf{Local} DNS Servers.
\end{enumerate}

\subsection{Risoluzione del nome}
La risoluzione del nome avviene leggendo il dominio da destra (ogni dominio dopo il \textit{TLD} ha un punto). Nel caso di ebay.it(.), per esempio: si legge da destra fino al primo ".". Nel primo passaggio la stringa ottenuta sarà vuota, per questo si interrogherà il Root DNS Server circa la stringa successiva, ovvero da dove si è arrivati fino al "." successivo: si è ottenuta la stringa ".it". Una volta ottenuta risposta dal Root DNS, sapremo chi è delegato dei domini associati al \textit{TLD} .it. A questo punto, continuando con lo stesso criterio, chiederemo al DNS Server ottenuto la stringa ottenuta successivamente, ovvero ".ebay". Procedendo in questo modo, si arriva alla risoluzione completa del nome. \\
Per questa procedura si possono usare due tipi di approcci:
\begin{itemize}
    \item \textbf{iterativo}:
    \begin{enumerate}
    	\item Il client chiede al nameserver locale la risoluzione del nome;
    	\item Il nameserver locale chiede e ottiene la risposta dal Root DNS;
    	\item Il nameserver locale chiede (l'informazione ottenuta nel passaggio precedente) e ottiene la risposta dal TLD DNS;
    	\item Il nameserver locale chiede (l'informazione ottenuta nel passaggio precedente) e ottiene la risposta dall'Authoritative DNS;
    	\item Ottenuta la risposta definitiva, la consegna al client.
    \end{enumerate}
    \item \textbf{ricorsivo}:
    \begin{enumerate}
    	\item Il client chiede al nameserver locale la risoluzione del nome;
    	\item Il nameserver chiede al Root DNS la risoluzione del nome;
    	\item Il Root DNS chiede al TLD DNS la risoluzione del TLD;
    	\item Il TLD DNS chiede all'Authoritative DNS la risoluzione del nome;
    	\item L'Authoritative DNS consegna la risposta.
    \end{enumerate}
\end{itemize}

\subsection{Cache}
I DNS utilizzano un parametro in ogni loro record, che ne gestisce il tempo di aggiornamento per il flush del record stesso, in modo tale da rendere l'aggiornamento periodico ogni tot tempo, e mantenerlo - nel restante - in cache, aumentando le prestazioni di risoluzione dei domini.

\newpage

\subsection{Records}
Il DNS conserva \textit{resource records} di vari tipi, tutti con il seguente formato: \{name, value, type, ttl\}.
I quattro formati più importanti:
\begin{itemize}
	\item \textbf{A}: 
	\begin{itemize}
		\item name: hostname
		\item value: IP
	\end{itemize}
	\item \textbf{NS}: 
	\begin{itemize}
		\item name: alias
		\item value: hostname server
	\end{itemize}
	\item \textbf{CNAME}:
	\begin{itemize}
		\item name: alias
		\item value: nome a cui punta l'alias
	\end{itemize}
	\item \textbf{MX}
	\begin{itemize}
		\item name: dominio
		\item value: hostname server
	\end{itemize}
\end{itemize}

\section{FTP}
L'\textit{FTP} (\textit{File Transfer Protocol}) è un protocollo utilizzato per il trasferimento di file da/a un host remoto. \\
Lavora sulla porta 21, ma ne usa - in fasi particolari anche un'altra:
\begin{itemize}
    \item \textbf{Porta 21}: porta standard per la connessione di controllo (persistente), che gestisce i l'autenticazione e un set di comandi per navigare nella directory FTP;
    \item \textbf{Porta 20}: porta - non persistente - utilizzata unicamente per l'effettivo trasferimento del file, che la chiude a trasferimento concluso.
\end{itemize}
Tutte le richieste FTP sono inviate come messaggi ASCII attraverso il canale di controllo.

\newpage

\section{Mail}
Composta da:
\begin{itemize}
    \item Agents (\textit{Thunderbird}, e così via \ldots);
    \item Servers (contenente \textit{mailboxes} e le code dei messaggi in uscita - \textit{message queues});
    \item Protocolli (\textit{SMTP}, \textit{POP}, \textit{IMAP}).
\end{itemize}
Ciascun protocollo gestisce funzioni diverse:
\begin{itemize}
    \item \textbf{SMTP} (in TCP, sulla porta 25, RFC 2821) (\textit{Simple Mail Transfer Protocol}): a questo protocollo è delegato il trasferimento delle emails. Segue tre fasi:
    \begin{enumerate}
        \item handshaking;
        \item transfer;
        \item closure.
    \end{enumerate}
    Avviene tutto tramite comandi formati da un ID e un messaggio (codificati in 7 bit). Mentre HTTP è un protocollo \textit{pull} (che richiede risorse), SMTP è un protocollo \textit{push} (che le invia).
    \item \textbf{POP} (in TCP, sulla porta 993, RFC 1939) (\textit{Post Office Protocol}): molto vecchio. Segue unicamente due fasi: l'autenticazione e la transazione (per il download delle email).
    \item \textbf{IMAP} (in TCP, sulla porta 143, RFC 3501) (\textit{Internet Mail Access Protocol}): mantiene ben organizzate tutte le emails sul server, fornendone  un'interazione molto più completa.
\end{itemize}


\section{Content Delivery Networks}
I \textit{CDN Providers} sono infrastrutture che forniscono caching dei contenuti web su larga scala, al posto dei \textit{WSP} (\textit{Web Service Provider}).
\paragraph{Esempio.}
\textit{Akamai} è uno dei \textit{CDN Provider} più famosi e importanti al mondo, e ha raggiunto nell'ultimo anno 200.000 server, distribuiti in 110 paesi. Si stima che abbia raccolto 30 terabit di dati e 2 trilioni di interazioni al giorno.
Le CDN vengono quindi utilizzate per migliorare il tempo di risposta percepito dal client portando i contenuti più vicini al network edge, quindi più vicino all'end-user. \\
\newpage

Sono composte delle seguenti componenti:
\begin{enumerate}
    \item \textit{Distribution System}: struttura effettiva di caching;
    \item \textit{Request Router}: struttura delegata per il routing da dominio al CDN più \textit{vicino} (non necessariamente a livello geografico, tendenzialmente per tempi di \textit{ping});
    \item \textit{Acounting-Billing}: struttura che gestisce le iscrizioni, i pagamenti e la configurazione degli IP/domini.
\end{enumerate}
Le CDN sfruttano un meccanismo di redirezione di due tipi:
\begin{itemize}
    \item \textbf{DNS Redirect}: a livello DNS, tramite l'associazione del record \textit{A} del dato dominio su un IP della CDN;
    \item \textbf{HTTP Redirect}: viene utilizzata un'URL apposita in cui specificata sia la \textit{redirezione} alla CDN, con il dominio di riferimento.
\end{itemize}

\section{P2P}
Nasce a fine anni '80 e diventa popolare con \textit{Napster}, chiuso da un ordinamento del 2001 per violazione di copyright. \\

\subsection{P2P networks}
Un tipo di network dove ogni peer può agire come client e come server e non deve essere sempre attivo: i peers entrano ed escono dalla rete continuamente. \\
L'idea di base fondamentale è l'aumento di scalabilità, di disponibilità di risorse, della privacy dei singoli peer e, infine, una drastica diminuzione dei costi. \\
Al tempo stesso, però, i peer non sono affidabili a causa della possibilità di disconnesione, sono eterogenei a livello di potenza computazionale, di rete e di storage, la ricerca/scoperta delle risorse all'interno della rete P2P spesso non è immediata, senza considerarne i problemi d'integrità. \\
Paragonate alle CDN, le reti P2P hanno diversi pro, a scapito della \textit{QoS} (\textit{Quality of Service}).

Sono spesso organizzate in maniera molto diversa tra loro.
\begin{itemize}
    \item \textbf{Rete non strutturata} (\textit{LimeWire}): non c'è nessuna regola. Una volta entrato nella rete, viene fornito un insieme di peer e stabilita una connessione. La ricerca di una risorsa avviene tramite \textit{flagging continuo}: se necessito di qualcosa vado dai peer vicini e la richiedo; se questi non la hanno essi stessi fanno la stessa richiesta ai peer a loro più vicini;
    \item \textbf{Rete strutturata} (\textit{KAD}, alcuni \textit{BitTorrent}): quando i peer si connettono vengono indicizzate le loro risorse in una \textit{DHT} (\textit{Distributed Hash Table}), che viene interrogata ogni qualvolta sia necessario ad un peer generico cercare una data risorsa, accellerandone i tempi di ottenimento, rispetto alle reti non strutturate. Tutto questo è però a scapito dei tempi di login nella / logout dalla rete;
    \item \textbf{Reti ibride}: reti istituite con la logica P2P, ma \textit{aiutate} tramite l'architettura client-server. Un set di server fornisce un indice centralizzato di risorse. Il problema foindamentale è che se questo set di server non fa parte più della rete - per qualsiasi ragione -, la rete non ha più senso di esistere;
    \item \textbf{Reti gerarchiche}: viene dato favore ai \textit{super-peer} (i peer migliori per capacità computazionali, di rete e di storage), ai quali si connette ogni altro peer.
\end{itemize}

\subsection{BitTorrent}
Sviluppato in Python nel 2001 da Bram Cohen.
Prevede un torrent separato per ogni file. I peer scaricano e caricano simultaneamente tutti i torrent. L'insieme di peer attivi è chiamato \textit{swarm}. Questi si dividono in \textit{seed}, che hanno il file completo, e in \textit{leechers}, che lo stanno ancora scaricando. \\
Gli utenti devono capire quali peers hanno una copia del file.
Una volta ottenuto un torrent, questo include un insieme di indirizzi IP, che puntano ad una categoria di server, chiamati \textit{tracker}, delegati di gestire le richieste di ciascun peer. Il \textit{tracker} consegna una lista di peer (~50 file) a cui potersi connettere. Questo non è assolutamente coinvolto con la reale distribuzione del file; è una lista di 50 peers al quale il client si connette via TCP. \\
Il \textit{tracker} inoltre mantiene informazioni sullo stato dei peer: ogni peer ogni 3 minuti inviano informazioni su loro stessi, circa il loro stato. Quando le connessioni di un peer agli originari 50 peer sono decrementate sotto le 20, viene fatta una nuova richiesta al \textit{tracker}, che sostituisce la lista attuale con una nuova. \\
Le porzioni di file che compongono il file che si ha intenzione di scaricare sono detti \textit{chunk} e hanno una dimensione che varia tra i 10KB a 1MB, anche se tendenzialmente è di 256KB. I \textit{chunk} non vengono scaricati a caso. Alla connesione, il client comincia per prima cosa a scaricare \textit{chunk} randomici, subito dopo quelli più rari (per il rischio di disconnessione e perdita definitiva e assoluta di quei \textit{chunk}). \\
Come il peer ottiene un \textit{chunk}, lo mette in upload ai peer collegati a lui: in base alla mia velocità di diffusione (in upload), vengo premiato in velocità di ricezione (in download).
L'upload usualmente viene indirizzato contemporaneamente a 5 peers:
\begin{itemize}
    \item 4/5: quelli da cui scarico più velocemente;
    \item 1/5: randomico, per evitare di provocarne il \textit{choking} (\textit{strozzamento}), perché altrimenti non verrebbe scelto mai.
\end{itemize}

\subsection{Spotify}
Servizio di streaming di musica on demand \textit{peer-assisted}.\\
Utilizza un protocollo proprietario. Dal 2014 ha cominciato a dismettere la sua rete P2P per il grande numero di utenti e lasciare solo CDN. \\
Utilizzava un metodo ibrido di distribuzione contenuti, facendo in modo tale che soltanto 8,8\% delle risorse venisse scaricato tramite i server CDN. \\
Per quanto riguarda la struttura P2P, era una rete di tipo non strutturato, con \textit{flagging} limitato a due \textit{hop} (due ricorsioni, nel routing della richiesta della risorsa ai peer vicini), un sistema semplice e funzionale attuabile grazie alla struttura (di \textit{fallback}) della CDN.

\paragraph{Caching}
Ogni canzone è criptata in una cache locale. Vantaggi: la canzone non va riscaricata e c'è più possibilità che un client riceva una traccia via P2P. Svantaggio: impatto sullo storage locale dell'utente. Per ovviare a questo problema, è applicata una \textit{Politica LRU} (\textit{least recently used}) per la pulizia della traccia. \\
Un client non può fare upload se non ha l'intera traccia. Ciò semplifica molto il protocollo: il client non deve comunicare con i peer di quali parti di traccia è in possesso.

\paragraph{Locating Peers}
Un tracker gestisce un insieme di peer. Ogni server è responsabile per una rete P2P di client distinta e indipendente. Per ogni traccia il server è a conoscenza solo degli ultimi 20 peer che l'hanno riprodotta. \\
In secondo luogo, ogni client è connesso ad un insieme di \textit{vicini} nella rete, ai quali può far richiesta di risorse tramite flagging - fermato automaticamente dopo 2 hop.
Un peer trasmette in upload ad un massimo 4 peers simultaneamente, attraverso chunk di dimensione fissa di 16KB. \\
Essendo \textit{Spotify} un servizio scritto per essere scalabile in maniera ottimale anche su piattaforme mobili, si è presupposto che una grande quantità di clienti facessero uso di rete mobile per accedere al servizio. Per questo è nata la necessità di gestire in maniera più pulita e funzionale possibile i limiti di banda. Esistono due limiti:
\begin{itemize}
    \item \textit{Soft}: il client non fa più richieste, ma risponde alle richieste di altri;
    \item \textit{Hard}: smette totalmente di aprire connessioni.
\end{itemize}

\subsubsection{Predicibilità delle tracce}
Circa il 61\% delle tracce sono riprodotte in un ordine predicibile, il restante 39 per cento in maniera casuale. \\
Quando le tracce sono riprodotte in maniere casuale, utilizzare la rete P2P sarebbe improponibile in termini di immediatezza di reperibilità del file della nuova traccia. Spotify utilizza in questo caso solo la CDN: questo significa più peso sulla sua rete - motivo per cui non riproducono più dei primi 15 secondi tramite quella connessione. I 15s sono sufficienti per determinare se l'utente ascolterà quella canzone o cambierà traccia, e quindi continuare a scaricarla - utilizzando la rete P2P -, o cambiarla. \\
Nel 92\% dei casi, quando un utente ascolta fino agli ultimi 30s di una canzone, ascolterà anche quella successiva. Per questo:
\begin{itemize}
    \item Mancando 20-30s: buffering via P2P;
    \item Mancando 10s (o skip track prima che finisca): buffering sulla CDN.
\end{itemize}
Il client monitora in continuazione il \textit{playout buffer}: se quest va sotto i 3 secondi va in emergenza, bloccando le trasmissioni in upload e richiedendo il file tramite CDN.

\section{Creazione di un sistema Client/Server}
Essendo i protocolli TCP e UDP implementati direttamente dal SO, implementare le opportune funzioni per interagire con essi è l'unica cosa necessaria per permettere l'invio e la ricezione dei dati.
I SO mettono a disposizione delle API, insiemi di funzioni per facilitare l'utilizzo di questi protocolli agli sviluppatori.
I \textbf{socket} sono fondamentalmente delle Internet API.
Un socket è una \textit{porta} che collega un'applicazione con un protocollo di trasporto; in Python è rappresentato da una struttura dati.
\begin{lstlisting}
import socket
s = socket.socket(addr_family, type, protocol)
\end{lstlisting}
Gli argomenti della funzione socket() specificano le informazioni necessarie per la creazione di un nuovo socket.
\begin{itemize}
\item \textbf{addr\_family}: famiglia di protocolli.
\begin{itemize}
	 \item socket.AF\_INET: IPv4;
	 \item socket.AF\_INET6: IPv6,
	 \item socket.AF\_UNIX: per gestire comunicazioni tra processi sulla stessa macchina fisica.
\end{itemize}
\item \textbf{type}: specifica se il tipo di comunicazione è orientata allo stream o a pacchetto (sostanzialmente, la differenza tra TCP e UDP):
\begin{itemize}
	\item socket.SOCK\_STREAM: stream-oriented (TCP);
	\item socket.SOCK\_DGRAM: message-oriented (UDP);
	\item socket.SOCK\_RAW: fornisce accesso al network layer.
\end{itemize}
\item protocol: se 0, specifica che il protocollo - in definitiva - da utilizzare è quello definito dalla tupla \textit{(addr\_family, type)}.
\end{itemize}

\subsection{TCP}
\begin{minipage}[t]{0.45\textwidth}
    \subsubsection{SERVER-SIDE}
    \paragraph{socket()}
    Definito nella pagina precedente.

    \paragraph{bind(\textit{address})}
    \textit{address} rappresenta una tupla \textit{(host,port)}, dove \textit{host} può essere un hostname o un indirizzo IP. Qualora \textit{host} fosse una strina vuota, si farebbe un \textit{bind} su tutte le interfacce di rete; invece, se fosse \textit{localhost}, lo si farebbe soltanto su quella di \textit{loopback}.
    \textit{port}: rappresenta il numero di porta su cui fare il \textit{bind}
	
    \paragraph{listen(\textit{backlog})}
    Rimane in attesa di richieste di connessione su quel socket. Il \textit{backlog} specifica il numero massimo di connessioni accodate (su Linux di default a 5, per motivi di sicurezza, come prevenzione da \textit{SYN flood attacks}). Se il \textit{backlog} è pieno, nuove connessioni possono essere ignorate o rifiutate.

    \paragraph{accept()}
    Preleva e stabilisce una connessione dal \textit{backlog}, creando un nuovo \textit{socket attivo}.

    \paragraph{send()}
    Descritto nel \textit{client-side}.

    \paragraph{recv(\textit{bufsize[, flags]})}
    Ritorna la stringa di dati ricevuti, delimitati da una quota massima, il \textit{bufsize}. In caso di assenza di dati, si blocca e il programma è in pausa fino all'arrivo di dati. Se invece torna una stringa vuota, significa che l'host dall'altro capo ha chiuso la connessione.
    \paragraph{close()}
    Descritto nel \textit{client-side}.
\end{minipage}
\hspace{8mm}
\begin{minipage}[t]{0.45\textwidth}
    \subsubsection{CLIENT-SIDE}
    \paragraph{socket()}
    Definito nella pagina precedente.

    \paragraph{connect(address)} Si connette ad un socket remoto. Il SO assegna automaticamente un indirizzo e un numero di porta.\\
    Argomenti: come server-side.
    Se è utilizzato un socket TCP, comunica al SO di iniziare la \textit{3-way handshake}.

    \paragraph{send()}
    Invia i dati in input, ritornando un intero che indica il numero di byte (rispetto alla lunghezza dell'input) effettivamente inviati. Possono verificarsi i seguenti casi:
    \begin{itemize}
        \item I dati sono correttamente ed interamente inviati;
        \item I dati sono parzialmente inviati (buffer parzialmente pieno): TCP deve spezzare i dati e ritorna soltanto la lunghezza in byte che è stata effettivamente inviata (il comando \textit{sendall} si assicura che venga inviato tutto);
        \item I dati non possono essere inviati (buffer pieno).
    \end{itemize}

    \paragraph{recv()}
    Descritto nel \textit{server-side}.

    \paragraph{close()}
    Chiude il socket: tutte le operazioni future sull'oggetto falliranno. La connessione non è chiusa immediatamente, perché il SO deve prima finire di inviare tutti i dati ancora presenti nel buffer.
\end{minipage}

\subsubsection{Chiusura di un socket}
Si è accennato all'utilizzo di una procedura \textit{close()}, che permette la chiusura - non propriamente immediata - di un socket \textit{attivo}. Qualora fosse però necessario definire in maniera più dettagliata il comportamento di un socket, o meglio, di una connessione relativa ad un socket, sarebbe più opportuno utilizzare il metodo \textit{shutdown()}. \\
Questo - a differenza del \textit{close()} - chiude uno dei due sensi di comunicazione, senza distruggere il socket:
    \begin{itemize}
        \item SHUT\_RD: chiude il canale server-client;
        \item SHUT\_WR: chiude il canale client-server;
        \item SHUT\_RDWR: chiude entrambi i canali.
    \end{itemize}

\subsubsection{Interfaccia di \textit{Loopback}}
Interfaccia network virtuale per gestire comunicazioni tra processi sullo stesso host. Bypassa l'hardware dell'interfaccia network locale e i livelli più bassi della pila TCP/IP

\subsubsection{Socket Passivo / Attivo}
Il socket di cui fino a questo punto si è parlato è il socket \textit{passivo}. Come accennato nelle procedure di implementazione di un socket in \textit{python}, si è parlato però dei socket \textit{attivi}. Questi si distinguono da quelli passivi per il fatto che sono effettivamente usati per la trasmissione dati. I primi sono creati da un processo, ma poi delegano a nuovi socket, con lo stesso \textit{address} (la tupla \textit{(indirizzo, porta)}), lo scambio effettivo dei dati. Un \textit{socket attivo} è definito univocamente dalla quadrupla \textit{(local\_ip, local\_port, remote\_ip, remote\_port)}.

\subsubsection{Porte}
I numeri di porta - essendo in \textit{16bit} - vanno dalla porta 0 alla 65535:
\begin{itemize}
    \item porte note: 0-1023 (dedicate, necessari i privilegi di root);
    \item porte registrate: 1024-49151 (utilizzate da servizi noti, non necessari però i privilegi di root);
    \item porte dinamiche: 49152-65535 (porte utilizzabili).
\end{itemize}
\textit{IANA} (o \textit{Internet Assigned Numbers Authority}) è l'organizzazione che si occupa di gestire la mappatura di associazioni \textit{porta-servizio}.

\subsubsection{Implementazioni utili}
\begin{lstlisting}
def recv_all(sock, len):
    data = ""
    while len(data) < len:
        read_data = sock.recv(len - len(data))
        if read_data == ``:
            raise EOFError('socket closed.')
        data += read\_data
    return data
\end{lstlisting}

\subsection{UDP}
In UDP - non esistendo connessione - si utilizza soltanto il comando di creazione socket, il \textit{bind()}, il \textit{sendto()} (che prende in input anche un nuovo parametro \textit{address}, che specifica la destinazione dei dati da inviare), il \textit{recvfrom()} (che da in output un nuovo parametro \textit{address} in riferimento al client inviante). \\
Si può usare la funzione \textit{connect()} con sockets UDP per evitare di specificare ogni volta l'indirizzo del server sul quale vogliamo chiamare una \textit{sendto()}. In questo modo il client non prevede la ricezione di pacchetti da altri mittenti. Si noti che l'utilizzo di \textit{connect()} non implica l'invio di dati. 

\paragraph{socket.settimeout(\textit{value})}
Imposta un timeout (in secondi) ricaricato ogni volta che si ricevono dati, al termine del quale il SO lancia una socket.timeout exception.