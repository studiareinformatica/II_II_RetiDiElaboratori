%P2P APPLICATIONS
\section{Paradigma P2P}
Nasce a fine anni '80 e diventa popolare ~ nel 2000 con Napster, chiuso da un ordinamento del 2001 per violazione di copyright.\\

\section{P2P networks}
Un tipo di network dove ogni peer può agire come client e come server e non deve essere sempre attivo: i peers entrano ed escono dalla rete continuamente.\\
\subsection{Scopo}
L'idea di base fondamentale è l'aumento di scalabilità, di disponibilità di risorse. 
Diminuisce i costi; aumenta la privacy dei peer.

\subsection{Challenges}
I peer non sono affidabili a causa della possibilità di disconnesione.
Scoperta delle risorse? Non è sempre facile.

\subsection{Flavours}
Sono spesso organizzate in maniera molto diversa tra loro.
\begin{itemize}
\item \textbf{Rete non strutturata}: (LimeWire) non c'è nessuna regola; entro nella rete, mi viene fornito un insieme di peer e stabilisco una connessione. Gestione semplice, richieste...un po' meno. Cercare una risorsa è fare flagging, se mi serve qualcosa vado dai peer vicini e la richiedo; se questi non la hanno essi stessi fanno la stessa richiesta ai peer a loro più vicini.
\item \textbf{Rete strutturata}: (KAD, alcuni BitTorrent) i peer sono connessi secondo una certa logica. Una DistributedHashTable permette una ricerca efficiente e distribuita con contenuti specifici. La 'pecca' di questo 
\item \textbf{Reti ibride}: Mescolano p2p e architetture client/server
\item \textbf{Reti gerarchiche}: Dotate di super-peer, individuati per quantità di banda / velocità in connessione. Ogni peer è conneso ad un super-peer che gestisce un indice locale delle risorse
\end{itemize}

\section{BitTorrent}
Sviluppato in Python nel 2001 da Bram Cohen.
Prevede un torrent separato per ogni file. I peer scaricano e caricano simultaneamente tutti i torrent. L'insieme di peer attivi è chiamato \textit{swarm}. Questi si dividono in \textit{seed}, che hanno il file completo, e in \textit{leechers}, che lo stanno ancora scaricando. \\
Gli utenti devono capire quali peers hanno una copia del file.
Una volta ottenuto un torrent, questo include un tracker con un insieme di peer (~50 file). Questo non è assolutamente coinvolto con la reale distribuzione del file; è una lista di 50 peers al quale il client si connette via TCP.

I chunk non vengono scaricati a caso. Alla connesione, il client tenta di scaricare per primi i chunk più rari (per il rischio di disconnessione).
Eccezione: random first. Quando un utente accede al torrent, i primi chunk sono selezionati casualmente.


%Alice seleziona top5 : 4 +rapidi in up e 1 random; ciò favorisce join di nuovi user nella rete. I peer unchoked sono quelli che riescono a scaricare, al contrario quelli choked.


\section{Spotify}
Servizio di streaming di musica on demand \textit{peer-assisted}.\\
Utilizza un protocollo proprietario. Dal 2014 ha cominciato a dismettere la sua rete p2p per il grande numero di utenti e lasciare solo CDN.
Utilizzava un metodo ibrido di distribuzione contenuti. Vantaggio principale: solo il 9 per cento circa dei dati veniva dai server spotify.

Rete non strutturata con flagging limitato, possibile grazie alla CDN. Vantaggi: mantiene semplice il protocollo, basssa la banda di clients, riduce latenza.

\subsection{Caching}
Ogni canzone è criptata in una cache locale. Vantaggi: la canzone non va riscaricata e c'è più possibilità che un client riceva una traccia via p2p. Svantaggio: impatto sul disco dell'utente. Politica LRU (least recently used) per pulizia della traccia.
Un client non può fare upload se non ha \textbf{l'intera traccia}. Ciò semplifica molto il protocollo: il client non deve comunicare con i peer di quali parti di traccia è in possesso.

\subsection{Locating Peers}
Un tracker gestisce un insieme di peer. Ogni server è responsabile per una rete p2p di client separata e indipendente. Per ogni traccia il server è a conoscenza solo degli ultimi 20 peer che l'hanno riprodotta. 
Secondo modo: ogni client è connesso ad un insieme di \textit{vicini} nella rete. Il flagging viene fermato automaticamente dopo 2 hop.
Un client fa upload a max 4 peers.
%...
Quando il limite \textit{soft} è raggiunto, il client inizia a pulire le connessioni per fare spazio a nuove in base al numero di byte trasmessi al peer %..
Circa il 61 per cento delle tracce sono riprodotte in un ordine predicibile, il restante 39 per cento in maniera casuale.

Quando le tracce sono riprodotte in maniere casuale, andare solo p2p sarebbe improponibile. Spotify utilizza in questo caso solo CDN; questo significa più peso sulla sua rete - quindi per questo non riproducono più dei primi 15 secondi tramite quella connessione. I 15s sono sufficienti per determinare se l'utente ascolterà quella canzone o cambierà traccia.\\

Quando le tracce sono riprodotte in maniera sequenziale: un utente che ascolta gli ultimi 30 secondi di una canzone, in 92/100 casi ascolterà la canzone successiva. 20 sec di buffering sulla p2p, in caso di fallimento (skip track) 10sec di buffering sulla CDN. 

Il client monitora in continuazione il playout buffer. Se va sotto ai 3 secondi va in emergenza. Stop uploading e usa CDN.

Le tracce sono divise in chunks di 16KB. Una traccia può essere contemporaneamente scaricata da CDN e p2p; se entrambe sono utilizzate, il client non scarica da CDN più di 15 secondi dopo il current playback point.