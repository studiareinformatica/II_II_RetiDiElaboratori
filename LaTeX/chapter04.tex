\section{Domain Name System}
Il \textbf{DNS} identifica sia il\textit{database distribuito}, che contiene l'associazione tra nome e IP, sia il protocollo applicativo per la risoluzione del db.\\
I suoi principali servizi sono:
\begin{itemize}
	\item traduzione tra hostname e IP
	\item gestione di aliasing dell'host
	\item gestione di aliasing del mail server
	\item distribuzione del carico (per evitare di sovraccaricare il server)
\end{itemize}

\subsection{Un database distribuito}

\begin{itemize}
	\item isolamento dei problemi
	\item gestione del traffico
	\item gestione delle distanze
	\item manutenzione
\end{itemize}

\textbf{Scalabilità}: capacità di un sistema di mantenere un comportamento ottimale alla crescita degli utenti che ne fanno uso.
Distribuito in maniera gerarchica.

\subsection{Risoluzione del nome}
Si può utilizzare un approccio iterativo o ricorsivo.
iterativo:
\begin{enumerate}
	\item si richiede un dominio al local dns server
	\item parsing della stringa
	\item parsing delle richieste della gerarchia
	\item ...
\end{enumerate}
ricorsivo:
\begin{enumerate}
	\item puts burden of name resolution on contacted name server
	\item heavy load at upper levels of hierarchy
\end{enumerate}

\subsection{Caching \& Updating}
Once any name server learns mapping, it \textit{caches} mapping.
Cache entries timeout after some time (\textbf{TTL});
TLD servers are typically cached in local name servers, thus root name servers are not often visited.
Cache entries may be out-of-date becaue of the 'best-effort' translate policy.

\subsection{Records}
Il DNS conserva \textit{resource records} di vari tipi. \\
Il loro formato è tipicamente \{name, value, type, ttl\}.
I quattro formati più importanti:
\begin{itemize}
	\item \textbf{A}: 
	\begin{itemize}
		\item name: hostname
		\item value: IP
	\end{itemize}
	\item \textbf{NS}: 
	\begin{itemize}
		\item name: alias
		\item value: hostname server
	\end{itemize}
	\item \textbf{CNAME}:
	\begin{itemize}
		\item name: alias
		\item value: nome a cui punta l'alias
	\end{itemize}
	\item \textbf{MX}
	\begin{itemize}
		\item name: dominio
		\item value: hostname server
	\end{itemize}
\end{itemize}