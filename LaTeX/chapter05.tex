Il livello di rete di occupa dell'effettivo trasporto delle informazioni dal sender al receiver, incapsulandole in datagrammi - lato sender - e consegnandoli al livello di trasporto - lato receiver.
Per la consegna, esamina il pacchetto, analizzando l'header e tutte le informazioni in esso contenute (principalmente l'IP), e determinandone l'instradamento più consono.

\paragraph{Forwarding}
Dirotta i pacchetti dall'input del router all'output dello stesso, associando un dato valore dell'header ad un link locale. \\\\
\paragraph{Routing}
Determina la rotta che collegherà la sorgente al destinatario, utilizzando appositi algoritmi di routing.\\\\

\section{Connection Setup}
Prima di verificare un percorso si deve verificare che ci sia tutto il necessario perché la trasmissione arrivi a compimento (che esistano le risorse necessarie, e così via \ldots). Quindi, prima che i datagrammi comincino a scorrere, i due \textit{end-hosts} e il router delegato aprono una \textit{connessione virtuale} dedicata. Inoltre, spesso sul livello direte vengono utilizzate delle funzioni importanti di terze parti, che definisco l'architettura di rete:
\begin{itemize}
	\item \textbf{ATM} (o \textit{Asynchronous Transfer Model});
	\item \textbf{frame-relay};
	\item \textbf{X.25}.
\end{itemize}
La differenza sostanziale tra il \textit{livello di rete} e il \textit{livello di trasporto} è che il primo mette in comunicazione due hosts (ed i vari router delegati, eventualmente), mentre il secondo due processi. \\
Il livello di rete può garantire diversi gruppi di garanzie:
\begin{itemize}
	\item Sul singolo datagramma:
	\begin{itemize}
		\item garanzia di consegna;
		\item garanzia di consegna entro un massimo di 40ms di ritardo.
	\end{itemize}
	\item Sull'intero flusso di datagrammi:
	\begin{itemize}
		\item consegna in ordine;
		\item garanzia di non superare la banda di flusso;
		\item garanzia di applicazione di restrizioni imposte nel corso della trasmissione.
	\end{itemize}
\end{itemize}
Internet (inteso come architettura di rete) non fornisce alcuna garanzia. \\
La rete a datagrammi però fornisce un servizio \textit{connectionless}, la cui alternativa è quella di utilizzare un approccio a \textit{circuito virtuale} utilizzato - per esempio - nell'architettura \textit{ATM} (determinando in una fase di setup il percorso da seguire - tabelle di routing - e mantenendo le informazioni di stato, che permettono di allocare risorse utili alla trasmissione).

\paragraph{VC (Virtual Circuit)}
Il sistema a \textit{VC} fa utilizzo di una tabelle di instradamento, che mette associazione ogni VC entrante - tramite un \textit{ID} univoco identificativo - ad un altro \textit{VC} uscente. \\
L'approccio a \textit{VC} viene utilizzato nelle architetture \textit{ATM}, \textit{frame-relay} e \textit{X.25}, ma non nell'architettura di internet attuale. \hfill \\

Nel caso di un approccio di tipo datagram - usato in internet -, non ho una fase di setup - e quindi nessun'informazione di stato -, e ogni pacchetto contiene tutte le informazioni  e le coordinate necessarie per la consegna. \\
La forwarding table cerca di compattare il maggior numero di informazioni nel minor spazio possibile, raggruppando gli IP per intervalli, e associando questi gruppi ad un dato link di uscita. Più concretamente, viene calcolato un \textit{longest address prefix} (e quindi cominciando con una dato prefisso binario - dove la sequenza binaria corrisponde all'IP): tutti gli IP che cominciando con lo stesso prefisso, vengono instradati verso lo stesso link.

\paragraph{Esempio.}
Ho una sequenza: \textit{11001000 00010111 00010110 10100001}. Leggendo questa sequenza bit per bit, controllo il più lungo prefisso che ha in comune con uno degli instradamenti descritti nella tabella di instradamento, dirottandolo verso il link associato a questo. \hfill \\

Quindi, dal router, vengono eseguiti degli algoritmi/protocolli di routing (\textit{RIP}, \textit{OSPF}, \textit{BHP}), per collegare il link d'entrata con quello d'uscita.

\section{Funzioni di un router}
\subsection{Input ports}
Viene gestita un buffer che contiene una coda dei pacchetti da instradare. In questa fase si utilizza (o si stabilisce) una tabella degli instradamenti, che mettono in comunicazione link di ingresso e di uscita tramite degli appositi switch.
\subsection{Switching fabrics}
Vari tipi di switch:
\begin{enumerate}
	\item Switching a memoria: prima generazione. 
	\item Switching a bus: ho \textit{n} linee di ingresso e \textit{n} linee di uscita, messe in collegamento tramite un bus condiviso. La pecca è che la velocità è limitata dall'architettura del bus. Inoltre, si può prelevare un datagramma alla volta, solamente per intero.
	\item Switching a rete di interconnessione (\textit{crossbar}): ho \textit{n} linee di ingresso e \textit{n} linee di uscita, interconnesse tra loro. In questo modo posso raggiungere tramite qualsiasi linea di ingresso, una qualsiasi linea di uscita. In questo caso i datagrammi possono essere prelevati divendoli a celle di lunghezza prefissata, dando all'architettura più elasticità.
\end{enumerate}
\subsection{Output ports}
Viene gestito un buffer dei datagrammi, in modo tale da gestire anche il caso in cui il tasso di trasmissione e ricezione dei datagrammi sia maggiore rispetto a quello di elaborazione dei datagrammi ricevuti.
\paragraph{HOL (Head-of-the-Line) blocking}
Blocco della coda dei pacchetti su un link d'ingresso, causato da un pacchetto non può essere ancora consegnato al suo link uscita, perché occupato a sua volta nell'elaborazione di altro pacchetto già inviato.

\section{Datagramma IP (IPv4)}
Grandezza minima header IP: 20byte. Campi:
\begin{enumerate}
	\item \textit{ver}: versione IP (IPv4 o IPv6);
	\item \textit{header length}: lunghezza dell'header in byte;
	\item \textit{type of service}: tipo dei dati, di datagramma inviato, base della gestione di servizi diversi attraverso la stessa struttura di datagrammi;
	\item \textit{time to live}: massimo numero di link che possono essere attraversati dal datagramma in questione;
	\item \textit{upper layer}: simile al numero di porta. Specifica a chi va consegnato a destinazione il datagramma, che sia un protocollo TCP o UDP o un altro tipo di protocollo arbitrario;
	\item \textit{16-bit indentifier, flgs, fragment offset}: parametri utilizzati per la frammentazione e la ricomposizione del datagramma una volta arrivato a destinazione;
	\item \textit{options}: campo opzionale, utile per tracciare, ottenere informazioni riguardo la rotta e/o la trasmissione generale del datagramma;
	\item \textit{32bit source IP}: IP sorgente;
	\item \textit{32bit destination IP}: IP destinatario;
	\item \textit{data}: dati trasportati.
\end{enumerate}

\subsection{IP Fragmentation / Riassembly}
I link di rete hanno l'MTU (la \textit{max transfer size}). Spesso occorre - per questa ragione - comprimere la grandezza di un datagramma, perché questo possa attraversare tutti i link. Per questa ragione di parla della fragmentation e del riassembly. Attraverso questi parametri viene indicato come è stato frammentato il datagramma (e in quanti sotto-datagrammi) e come poter riassemblarlo una volta giunto a destinazione. Ciascun sotto datagramma ha in comune il parametro identificativo di 16bit e tutti, ad esclusione dell'ultimo pacchetto derivato dalla frammentazione, hanno il parametro \textit{fragflag} settato ad \textit{1}.

\section{Indirizzo IP}
Ogni indirizzo IP viene ad essere associato ad una porta di rete. Infatti, il router, avendo più porte di rete, ha un diverso indirizzo IP per ciascuna. \\
L'indirizzo IP è una sequenza a 32bit, che viene divisa in 4 sequenze da 8 bit, ciascuna corrispondente al valore decimale delle 4 componenti dell'indirizzo IP comunemente utilizzato: \textit{1.1.1.1} : \textit{00000001 00000001 00000001 0000001}.

\subsection{Subnet}
La subnet è la sottorete che mette in comunicazione tutti gli indirizzi IP con \textit{subnet part} (sequenza\textit{high order bits}) in comune.
\paragraph{CIDR (Classless InterDomain Routing)}
Approccio di rappresentazione di subnet tramite la rappresentazione dei bit della \textit{subnet part}, alla quale viene aggiunta la \textit{host part} settata a zero. Infine si aggiunge un parametro che indica la capienza totale della subnet: \textit{192.168.1.0/24} (subnet \textit{192.168.1}, che può contenere un massimo di 255 \textit{host part} diverse, e quindi 255 indirizzi IP univoci). \hfill \\
\paragraph{DHCP (Dynamic Host Configuration Protocol)}
Protocollo di configurazione e associazione automatica dell'indirizzo IP ad un dato host in connessione ad una data subnet.
Sostanzialmente l'host è capace in questo modo di ricevere dal server di rete un indirizzo valido dinamicamente, così da potersi unire alla rete.
Il DHCP può fornire anche il \textit{first-hop router} (\textit{gateway} della subnet) e il DNS server. \\
Le richieste DCHP sono incapsulate in \textit{UDP}. \\
Questo permette il riutilizzo di indirizzi (che vengono tenuti occupati limitatamente al tempo entro il quale sono effettivamente connessi).

\section{ICMP}
Per comunicare informazioni a livello di rete, hosts e routers utilizzano l'\textit{Internet Control Message Protocol}. Tali informazioni includono report di errori (host irraggiungibile, network, porte, protocolli) e richieste/risposte echo (usate da ping).\\
Tali messaggi risiedono nei datagrammi IP e sono composti da tipo, codice e i primi 8 byte del datagramma IP che ha causato l'errore.
\begin{center}
	\begin{tabular}{|c|c|p{5cm}|}
		\hline
		Type & Code & Description\\
		\hline
		0 & 0 & echo reply (ping)\\
		\hline
		3 & 0 & dest. network unreachable\\
		\hline
		3 & 1 & dest. host unreachable\\
		\hline
		3 & 2 & dest. protocol unreachable\\ 
		\hline
		3 & 3 & dest. port unreachable\\
		\hline
		3 & 6 & dest. network unknown\\
		\hline
		3 & 7 & dest. host unknown\\
		\hline
		4 & 0 & source quench (congestion control - not used)\\
		\hline
		8 & 0 & echo request (ping)\\
		\hline
		9 & 0 & route advertisement\\
		\hline
		10& 0 & router discovery\\
		\hline
		11& 0 & TTL expired\\
		\hline
		12& 0 & bad IP header\\
		\hline
	\end{tabular}
\end{center}

\paragraph{Traceroute e ICMP}
Il mittente invia una serie di segmenti UDP a destinazione, il primo con TTL = 1, il secondo con TTL = 2 e così via.
Quando l'\textit{n}-esimo set di datagrammi arriva all'\textit{n}-esimo router, il router scarta i datagrami e invia i messaggi ICMP del mittente. Una volta arrivati, il mittente registra gli RTT.\\
Per terminare il processo, quando tutti i segmenti UDP arrivano a destinazione, il destinatario restituisce un ICMP di tipo 3, codice 3: \textit{port unreachable}. A questo punto, il mittente interrompe l'invio.
\section{IPv6}
Il motivo iniziale dell'utilizzo di questa versione di IP è il fatto che lo spazio di un indirizzo da 32 bit impiega poco tempo per essere completamente allocato. \\
% Additional motivation: no time for translating! 
% - header format helps speed processing/forwarding
% - header changes to facilitate QoS
Il formato dei datagrammi IPv6 prevede un header a dimensione fissa di 40 byte e impedisce la frammentazione.
\begin{itemize}
	\item \textbf{pri}: indica la priorità tra datagrammi nel flusso
	\item \textbf{flow label}: indica datagrammi nello stesso "flusso"
	\item \textbf{next header}: indica il protocollo del livello più alto per i dati%?
\end{itemize}
%includere immagine
Altri cambiamenti rispetto a IPv4 includono:
\begin{itemize}
	\item la \textbf{checksum} è rimossa interamente per ridurre il tempo di processamento ad ogni hop;
	\item le \textbf{opzioni} sono consentite, ma al di fuori dall'header, e indicate dal campo \textit{next header};
	\item \textbf{ICMPv6}, una nuova versione di ICMP, include tipi di messaggi aggiuntivi (come \textit{"Packet Too Big"}), e funzioni di gestione multicast di gruppo.
\end{itemize}

\paragraph{Transizione da IPv4 a IPv6}
Non tutti i routers possono essere aggiornati contemporaneamente. Perchè il network possa operare contemporaneamente con router IPv4 e IPv6, è introdotto il \textbf{tunneling}: i datagrammi IPv6 sono trasportati come \textit{payload} nei datagrammi IPv4 per i router che li richiedono.
%controllare slide + cercare appunti: schemi tunneling, "interplay between routing,forwarding", "graph abstraction".
\section{Algoritmi di routing}
La scelta dell'algoritmo di routing più adatto varia in base a due domande fondamentali:

\paragraph{1. Informazione globale o decentralizzata?}
\begin{itemize}
	\item Globale: tutti i router hanno completi "topology, link cost info"%?
	. Per questa scelta è consigliato l'utilizzo di algoritmi \textit{link state}.
	\item Decentralizzata: i router conoscono i vicini fisicamente connessi, "link costs to neighbors"; è previsto un processo iterativo di computazione per lo scambio di informazioni con i vicini. Per questa scelta è consigliato l'utilizzo di algoritmi \textit{distance vector}.	
\end{itemize}

\paragraph{2. Statico o dinamico?}
\begin{itemize}
	\item Statico: il router cambia lentamente nel tempo.
	\item Dinamico: il router cambia più rapidamente e periodicamente in risposta a cambiamenti di "link costs".
\end{itemize}

\section{Tabella degli instradamenti}
\subsection{Algoritmo di Dijkstra}
L'algoritmo di Dijkstra è un algoritmo routing \textit{link state} che prevede che i costi di linking siano conosciuti da tutti i nodi. Ciò è ottenuto grazie al \textit{link state broadcast}, grazie a cui tutti i nodi avranno le stesse informazioni.\\
Questo algoritmo calcola i percorsi di costo minimo da un nodo a tutti gli altri, producendo una \textit{forwarding table} per quel nodo. È iterativo: dopo \textit{k} iterazioni fornisce il percorso di costo minore per k destinazioni.
\paragraph{Notazione}
\begin{itemize}
	\item $ c(x,y) $: costo di link dal nodo x al nodo y; corrisponde a $ \infty $ se non sono vicini diretti.
	\item  $ D(v) $: valore corrente del costo di percorso dal nodo di partenza al nodo di destinazione \textit{v}.
	\item  $ p(v) $: nodo predecessore nel percorso dal nodo di partenza al nodo \textit{v}.
	\item  $ N' $: set di nodi il cui minore costo di percorso è conosciuto.
\end{itemize}

\begin{algorithm}
	\caption{Algoritmo di Dijsktra}\label{alg:dijsktra}
	\begin{algorithmic}
		\Function{Dijsktra}{}
		\State $ N' = {u} $
		\For {v $ \in $ net }
		\If {v.isAdjacent(u)}
		\State $ D(v) = c(u,v) $
		\Else $ D(v) = \infty $
		\EndIf
		\EndFor
		\While {$ N'.isFull() $} \Comment{Finchè N' non contiene tutti i nodi}
		\State $ find (w : (!N'.contains(v)) \and (D(v).isMinimum())) $
		\State $ N' \gets w $
		\State $ D(v) = min(D(v),D(w)+c(w,v)) $
		\Comment{Il nuovo costo per v è uguale al vecchio costo
			o uguale al costo del percorso minimo per w più il costo da w a v}
		\EndWhile
		\EndFunction
	\end{algorithmic}
\end{algorithm}

\paragraph{Correttezza}

\subparagraph{Tesi:} L'algoritmo di Dijkstra è corretto se eseguito su un grafo pesato diretto $ G = (N,E) $ con pesi non negativi, sorgente u, funzione peso c, allora alla terminazione $ D(v) = \delta(u,v) $ per ogni nodo \textit{v} in \textit{N}, dove $ \delta(u,v) $ è la lunghezza del cammino di peso minimo tra \textit{u} e \textit{v}.

\subparagraph{Dimostrazione:}
$ D(v) $ nonè più aggiornato nel momento in cui v è inserito in N'. Dovremo quindi mostrare che $ D(v) = \delta(u,v) \forall v$. \\
Ragioniamo per assurdo: sia x il primo nodo (nell'ordine di inserimento in N') per cui vale $ D(v) \neq \delta(u,v) $ al momento in cui x è inserito nell'insieme N'. \\
Ne consegue che $ x \neq u $, essendo il nodo sorgente u inserito nella fase di inizializzazione e valendo $ D(u) = \delta(u,v) = 0 $. Deve inoltre esistere un percorso di costo non infinito da u a x, dato che altrimenti il valore a cui D(x) è inizializzato (cioè $ \infty $) sarebbe uguale a delta(u,v). Esiste quindi un percorso di costo minimo $ p=u \ldots v \to y \ldots x  $, dove y è il primo nodo sul percorso di costo minimo \textit{non} in N' (quindi i nodi u \ldots v sono \textit{tutti} in '). Il percorso p può quindi essere diviso in due percorsi: $ p_{1} $, che va da u a y, e $ p_{2} $, che va da y a x. \\
Da notare che il percorso $ p_{1} $ è anch'esso il percorso di costo minimo che unisce u a y: se non lo fosse, e ci fosse un percorso $ p_{3} $ che unisce u a y di costo minore del costo di $ p_{1} $, allora la concatenazione di $ p_{2} $ e $ p_{2} $ sarebbe un percorso p' da u a x di costo minore di p, ma ciò sarebbe assurdo a causa dell'assunto iniziale secondo cui p è un percorso minimo. \\
Quando x è inserito in N', $ D(y) = \delta(u,y) $. \\
Infatti, in quel momento v è stato già inserito in N' e dopo il suo inserimento y ha ricalcolato $ D(y) = D(v)+c)(v,y) = \delta(u,v)+c(v,y) $, dato che per ipotesi x è il primo nodo per cui all'inserimento in N' la stima dei costi non corrisponde al percorso di costo minimo $ \delta(u,y) $. \\
Dato che y precede x sul percorso minimo ed i pesi sugli archi sono non negativi, vale che: \\
$ \delta(u,x)\geq\delta(u,y)=D(y) $\\
e quindi anche \\
$ D(x) \geq \delta(u,x) \geq \delta(u,y)=D(y) $. \\\\

D'altro canto, dato che x viene inserito in N' prima di y, vale che: \\
$ \delta(u,x) \leq D(x) \leq D(y) = \delta(u,y) $; \\
quindi \\
$ \delta(u,x)=D(x)=D(y)=\delta(u,y) $, \\
cosa che porlta alla contraddizione. \\\\

\subparagraph{Discussione}
La complessità dell'algoritmo per \textit{n} nodi, contando che ogni iterazione necessita di controllare tutti i nodi \textit{w} non in N, è $ \frac{n(n+1)}{2} \approx O(n^2) $. Il costo della più efficiente implementazione possibile è $ O(nlogn) $.

\subsection{DistanceVector}
Uno degli algoritmi alternativi a quello di \textit{Dijsktra}, è noto come algoritmo di \textit{Bellman-Ford}. Questo dice: essendo $d_{x}(y)$ definito come il percorso di costo minimo da \textit{x} a \textit{y}, allora $d_{x}(y) = min\{c(x,v), d_{v}(y)\}$, dove il \textit{min} indica il minimo preso per ciascun vicino comune tra il \textit{v} e quello \textit{x}, \textit{c(x,v)} è il costo per arrivare dal nodo \textit{x} a quello \textit{v} e $d_{v}(y)$ è il costo dal nodo \textit{v} a quello \textit{y}. In quasto caso il raggiungimento con percorso minimo di un nodo è il successivo hop nel percorso minore, usato nella tabella degli instradamenti. \\
Altro modo di rappresentare questa equazione è:
\begin{center}
	$d_{x}(y)^{h+1} = min\{d_{x}(y)^h, min\{c(x,v) + d_{v}(y)^h\}\}$
\end{center}
Quindi: devo andare da \textit{x} a \textit{y}, passando al più per \textit{h} o esattamente \textit{h+1} uscite. \\
La potenza dell'equazione di \textit{Bellman-Ford} è che non è richiesta sincronizzazione tra i vari nodi all'interno di una rete. \\
Genericamente, ogni nodo tiene un \textit{distance vector} che memorizza tutti i costi del nodo verso i suoi vicini. Ogni nodo invia il suo \textit{distance vector} ai vicini, che quando lo ricevono aggiornano il proprio utilizzando l'equazione di \textit{Bellman-Ford}, per ogni nodo. Sotto condizioni naturali, la stima della distanza a costo minimo calcolata con questo sistema converge in quella realmente tale. \\
Dunque: questo sistema è \textit{iterativo}, \textit{asincrono} e \textit{distribuito}. Non esiste né un inizio né una fine dell'algoritmo, perché per ogni cambiamento minimo nel \textit{distance vector} di un nodo, questo viene propagato e notificato ad ogni suo vicino. \\
Il punto di debolezza di questo sistema è il \textit{count to infinity}, cioè la lentezza nel ricalcolare le stime migliori al cambio di anche un solo costo: per ogni minimo cambiamento - come l'eventuale scomparsa di un collegamento -, sono necessarie altre nuove \textit{n} iterazioni prima di ritrovarci in una situazione stabile ed ottimale.

\paragraph{Poisoned reverse}
Se il nodo \textit{z} decide di voler passare per \textit{y} per arrivare ad \textit{x}, \textit{z} comunica a \textit{y} che la distanza che separa \textit{z} da \textit{x} è pari ad infinito, così da permettere l'instradamento per \textit{y}. Questo però non permette totalmente di risolvere i problemi di \textit{count-to-infinity}.

\subsubsection{Costi e conclusioni}
Una volta calcolate le tabelle dei cammini minimi, popolo la tabella degli instradamenti, che sappiamo avere:
\begin{itemize}
	\item Complessità dei messaggi:
	\begin{itemize}
		\item \textit{LinkState}: dati \textit{n} nodi ed \textit{E} links, ho un numero di messaggi inviati pari a $O(n\times E)$;
		\item \textit{DistanceVector}: variabile, in quanto - per ciascun nodo - avvengono scambi e interazioni con i suoi vicini.
	\end{itemize}
	\item Robustezza:
	\begin{itemize}
		\item \textit{LinkState}: dati \textit{n} nodi, ho una velocità di $O(n^2)$; potrebbe portare oscillazioni;
		\item \textit{DistanceVector}: variabile; potrebbe portare a loop o al problema di \textit{count-to-infinity}.
	\end{itemize}
\end{itemize} \hfill \\
Tutte queste soluzioni di instradamento vengono utilizzate in un solo \textit{AS}.

\section{Autonomous System (AS)}
Sistema di aggregazione di router in diverse regioni, governato da un entità - il \textit{gateway router} - tramite un protocollo di instradamento \textit{intra-as} (variabile in base all'\textit{AS} preso in considerazione).
Esistono poi i protocolli \textit{inter-as}, che si occupano di mettere in connessione differenti \textit{AS} tramite \textit{router di frontiera}. \\
\begin{itemize}
	\item L'\textit{intra-as} avrà la gestione degli instradamenti interni all'\textit{AS}, per avere informazioni sul percorso minimo per raggiungere un nodo \textit{interno} all'\textit{AS}: si appoggia sulla necessità di presenza di criteri per rendere la gestione degli indirizzamenti il più organizzata possibile, spesso anche a scapito della performance;
	\item L'\textit{inter-as} avrà la gestione degli instradamenti esterni all'\textit{AS}, per ottenere informazioni sui percorsi utilizzati per mettere in comunicazione router appartenenti ad \textit{AS} differneti: si appoggia sulla necessità di performance.
\end{itemize}

\paragraph{Esempio}
Mi trovo in un \textit{AS}, l'\textit{AS1}. Ricevo un datagramma destinato ad un nodo esterno all'\textit{AS1}. Devo inviarlo al gateway, ma quale? L'\textit{AS1} deve poter apprendere quali sono le destinazioni raggiungibili tramite ciascun suo \textit{AS} vicino, e inoltre deve dare le sue informazioni delle destinazioni raggiungibili tramite l'\textit{AS1} agli stessi, cosi da risultare raggiungibile per ricevere eventuali risposte indietro. \\
Tramite \textit{inter-as} impara che una subnet \textit{x} è raggiungibile tramite \textit{AS3}, per esempio, uscendo col \textit{gateway router} \textit{1c}: viene impiegato l'\textit{inter-as} per propagare le informazioni di raggiungibilità e calcolare il percorso migliore. \\
Ma se invece potessi apprendesse di poter raggiungere la destinazione sia tramite \textit{AS3} che tramite \textit{AS2}? Il router \textit{1d} deve determinare quale dei due sia migliore da adoperare: questa scelta spetta comunque all'\textit{inter-as}. Molto spesso si usa scegliere il più vicino (metodo \textit{hot potato routing}).

\subsection{Protocolli di instradamento intra-as}
Protocolli noti anche come \textit{Interior Gateway Protocols} (\textit{IGP}). Tra i più famosi:
\begin{itemize}
	\item \textit{RIP}, o \textit{Routing Information Protocol};
	\item \textit{OSPF}, o \textit{Open Shortest Path First};
	\item \textit{IGRP}, o \textit{Interior Gateway Routing Protocol}.
\end{itemize}

\paragraph{RIP}
Protocollo distribuito da \textit{BSD-Unix} nel 1982, che utilizza un approccio di tipo \textit{distance vector}. \\
Per calcolare il costo di raggiungibilità, conta il numero di \textit{subnet} (numero di hop, il cui massimo imposto è 15, cosi da avere un metodo rapido per stabilire se una rotta è convenzionalmente ritenuta infinita o meno) da attraversare per arrivare a destinazione. Ad ogni di 30 secondi, viene inviato un \textit{advertisement} (un semplice messaggio) sui costi attuali relativi a ciascun nodo: in questo tipo di messaggio posso comunicare al massimo 25 \textit{subnet}.
Qualora non si ricevesse nemmeno un \textit{advertisement} di un dato nodo entro 180 secondi, quel nodo verrebbe definitivamente dichiarato dai vicini come morto: le rotte verrebbero invalidate. \\
Le tabelle di \textit{RIP} vengono gestite a livello applicativo, e gli advertisement sono inviati tramite protocollo \textit{UDP}.

\paragraph{OSPF}
Approccio completamente \textit{open}, molto più recente rispetto al \textit{RIP}, ma di tipo \textit{link state}: è necessaria una fase di inizializzazione in cui ogni nodo deve comunicare in \textit{broadcast} a tutti i nodi le sue informazioni. \\
Per comunicare le informazioni - tramite advertisement - occorre autenticarsi. \\
Dà la possibilità di individuare più di un percorso ottimale e di conservarlo per essere utilizzato per trasmettere dati, permettendo la gestione del \textit{load balancing}: tramite questa funzionalità, supporta non solo l'\textit{unicast} (la trasmissione biunivoca sorgente-destinatario), ma anche il \textit{multicast} (permette di trasmettere da un'unica sorgente ad un gruppo di destinazione). \\
Non è \textit{flat}, ma gerarchico, così da ovviare a problemi derivanti dalla grandezza di un'eventuale struttura: si divide in \textit{n}, ciascuna gestita da un \textit{router di bordo}, che - oltre a catalogare informazioni di instradamento e raggiungibilità relative a quell'area - si trova in un'area speciale, detta \textit{area di backbone}, che mette in comunicazione tutti i \textit{router di bordo} con il \textit{router di backbone}. 

\subsection{Protocolli di instradamento inter-as}
Senza ombra di dubbio, il protocollo \textit{de facto} \textit{inter-as} più famoso e utilizzato è il \textit{BGP} (o \textit{Border Gateway Protocol}).

\paragraph{BGP}
Protocollo che non abbraccia propriamente né l'approccio \textit{link state}, né quello \textit{distance vector}. Definisce il suo approccio come \textit{path vector}, in quanto usa dichiarare unicamente l'insieme di vettori che mettono in comunicazione diversi \textit{AS}. \\
Utilizza due componenti:
\begin{itemize}
	\item \textit{eBGP}: per comunicare ai diversi \textit{AS} la raggiungibilità di destinazioni possibili tramite il mio \textit{AS};
	\item \textit{iBGP}: per propagare la raggiungibilità a tutti i router interni all'\textit{AS};
\end{itemize}
Due router \textit{BGP} (due \textit{peers}), per scambiarsi messaggi \textit{BGP} (atti alla scoperta dei percorsi possibili per raggiungere diversi prefissi di rete), richiedono l'apertura di una conessione \textit{TCP}, una sessione dedicata. \\
Inoltre, con \textit{BGP}, quando un \textit{AS} comunica la raggiungibilità di un dato prefisso di rete, prende l'impegno di poter gestire il traffico derivato dalla richiesta di connessioni vesro quella destinazione. \\
Infine, utilizza due attributi fondamentali:
\begin{itemize}
	\item \textit{AS-PATH}: contiene gli \textit{AS} attraverso cui è passato l'advertisement del prefisso di rete;
	\item \textit{NEXT-HOP}: indica il router interno specifico sull'\textit{AS} di \textit{next-hop} (vicino, adiacente).
\end{itemize}