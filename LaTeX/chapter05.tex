\section{FTP}
Protocollo per il trasferimento di file da/a un host remoto.
Utilizza di base due connessioni. Ogni volta che un utente vuole richiedere un file, contatta la porta 21. Una volta ricevuta l'autorizzazione, il client permette di navigare nella directory remota ed inviare comandi. \\
Quando il server riceve un comando di trasmissione file, apre una seconda connessione dati TCP \textit{verso} il client.È un tipo di connessione \textit{non persistente}: una volta conclusa la trasmissione, il server chiude la connessione. \\
È utilizzato un controllo di tipo applicativo; le richieste sono inviate come ASCII attraverso il canale di controllo.
Tipici comandi sono \textbf{user}, \textbf{pass}, \textbf{list} \(che restituisce una lista di file nella cwd\), \textbf{retr} + filename, \textbf{stor/put} + filename. 



\textbf{Stateful}: perchè se deve memorizzà na cartella, però non ho capito che ce sta scritto.

\section{SMTP}
Uses TCP to reliably transfer email message from cli to serv through port 25.
Direct transfer %...

SMTP uses persistent connections.
Requires message to be 7bit ASCII.
server uses CRLF.CRLF to determine end of message

Comparison w/ http:
%...

Mail access protocols:

SMTP delivery/storage to receiver's server
mail acces protocol: retrieval from server
3possible ways:
	- POP: auth, download \(old but gold\)
	- IMAP: more features, including manipulation of stored msgs on servers
	- HTTP: gmail, yahoo etc.

IMAP keeps all messages at server; allows user to organize messages in folders \& keeps user state across sessions - so keeps names of folders + mappings between msg IDs and folder name