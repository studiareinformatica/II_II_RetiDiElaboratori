\section{Protocolli di Trasporto}
I protocolli di trasporto sono un insieme di protocolli che risiedono sugli host e sugli endsystems; si occupano di fornire le regole per una comunicazione logica tra due entità remote.\\
A livello di invio, si occupano della ricezione dei messaggi in segmenti e li inviano al network layer; alla ricezione, i segmenti sono riassemblati in messaggi e passati al livello applicativo.\\
Mentre il livello di rete offre comunicazione tra due endsystems (trasportando informazioni da un host all'altro), il livello di trasporto crea un collegamento logico tra due \textit{processi}. Deve quindi essere in grado di raccogliere flussi informativi, utilizzare il livello di rete per trasportarli e successivamente smistarli nei vari processi applicativi.

\section{Internet transport-layer protocols}
A dispetto del fatto che il livello rete utilizza una best-effort policy che non dà nessuna garanzia di consegna, \textbf{TCP} si occuperà di applicare meccanismi per la traduzione di informazioni agli opportuni processi anche nel caso di perdita di dati.\\
I principali tipi di perdita dei dati sono:
\begin{itemize}
	\item un eccesso di informazioni arrivate al router intermedio. \\
	La conseguenza è un riempimento delle code fino al raggiungimento di una congestione in rete e all'inizio di una perdita di pacchetti.
	\item il ricevimento a destinazione di un tasso di informazioni troppo elevato rispetto alla velocità di lettura. \\
\end{itemize}
Si noti che un controllo del flusso di informazioni richiede algoritmi diversi rispetto al controllo di congestione; entrambi i tipi di controllo sono implementati da TCP.

\section{Multiplexing e Demultiplexing}
L'invio di informazioni da parte di un processo è eseguito attraverso una determinata porta dotata di interfacce ben definite. Tale porta, che consente il colloqui con l'entità sottostante di livello trasporto, prende il nome di \textit{socket}.\\
Poiché più flussi informativi sono trasmessi tramite lo stesso canale logico, occorre stabilire delle funzioni di smistamento di dati. Queste funzioni sono implementate nei protocolli TCP e UDP in maniera differente.\\

\subsection{UDP: Connectionless demux}
Il socket creato è dotato di un numero di porta dell'host locale; quando viene creato un \textit{datagram} (pacchetto) da inviare nel socket UDP, questo deve specificare l'indirizzo IP e il numero di porta dell'host di destinazione.
Quando l'host remoto riceve un segmento via UDP, controlla il suo numero di porta di destinazione e lo indirizza alla porta corrispondente.
Pacchetti con stesso numero di porta di destinazione, ma indirizzi IP o porte di origine differenti saranno quindi orientati allo stesso socket a destinazione.

\section{TCP: Connection-oriented demux}
In questo caso viene creato un socket dedicato al flusso di informazioni identificato dalla presenza di quattro campi: due per gli indirizzi IP del mittente e del destinatario, due per i numeri di porta del mittente e del destinatario.
Nel caso di più client connessi allo stesso webserver, all'inizio viene fatta una richiesta su un \textit{welcome-socket} generale; successivamente le richieste sono smistate ad un socket dedicato su ciascuna delle connessioni: in questo modo è assicurato che il flusso di dati passerà per socket differenti.