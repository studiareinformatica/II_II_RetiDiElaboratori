Come si trasmettono le informazioni da un router all'altro fino agli endpoint è opera del livello \textit{datalink}. \\
Questo livello si appoggia sul concetto di \textit{trama}, un percorso seguito dal flusso, che inizia e finisce su due estremi. Offre diversi servizi:
\begin{itemize}
	\item rilevamento e correzioni di errori;
	\item condivisione di canali broadcast, offrendo accessi multipli;
	\item servizio di gestione degli indirizzi di livello link;
	\item trasnferimento dati affidabili, con controllo del flusso.
\end{itemize}
Il livello di rete è implementato sugli \textit{adaptor}, che comunicano l'uno con l'altro - stabilendo una trama per raggiungersi - e che devono rispettare gli standard appena esposti. \\
Accetta soltanto degli stream di raw bit (flussi di sequenze di bit) e cerca di consegnarlo a destinazione: ciononostante la comunicazione non è necessariamente priva di errori. Per gestire un numero multiplo di flussi di informazioni, divide lo stream in un numero variabile di trama (\textit{framing}) e elabora il checksum per ogni trama (per ovviare a problemi di generazione e correzione errori). \\
Lo split viene applicato ogni \textit{n} caratteri, e ciascun flusso è delimitato da dei caratteri speciali: \textit{DLE STX} (\textit{Data Link Escape Start of TeXt}) e DLE ETX (\textit{Data Link Escape End of TeXt}). Qualora dovesse essere inviata esattamente la stringa utilizzata come escape, viene aggiunto un \textit{DLE} per eliminare ambiguità. \\
Possono essere poi utilizzati degli approcci meno generici, come il sistema \textit{Manchester}.

\section{Rilevamento di errori}
Tra i dati trasmessi, vi sono due blocchi fondamentali:
\begin{enumerate}
	\item \textit{EDC}: bit - ridondnati - di \textit{error detection and correction};
	\item \textit{D}: dati protetti dal controllo di errori - eventualmente accompagnati da campi header.
\end{enumerate}
Ovviamente la rilevamento di errori non è affidabile al 100\%: si potrebbero raramente non rilevare alcuni errori, in maniera inversamente proporzionale alla dimensione alla dimensione dell'\textit{EDC} utilizzata. \\
Nel rilevamento di errori, si utilizza il calcolo della distanza di \textit{Hamming}, attraverso cui è possibile determinare in quanti bit differiscano due parole di codice. Sostanzialmente, si esegue lo \textit{XOR} tra le parole e poi si conta il numero di 1 nel risultato: il numero di posizioni nelle quali le due parole di codice differiscono determina la loro distanza di \textit{Hamming}. Se due parole hanno distanza di \textit{Hamming} \textit{d}, ci vorranno \textit{d} operazioni sui singoli bit per tramutare una parola di codice nell'altra: per come sono utilizzati i bit di ridonanza, se la lunghezza delle parole di codice è $n=m+4$, sono possibili $2^m$ messaggi dati validi, ma non tutte le $2^n$ parole di codice. Dunque la distanza di \textit{Hamming} di un codice è la minima distanza di \textit{Hamming} tra le due parole di codice. Ancora, per rilevare \textit{d} errori occorre un codice con distanza di \textit{Hamming} $d+1$, mentre per correggerli serve un codice con distanza di \textit{Hamming} $2d+1$.

\subsection{CRC (Cyclic Redundancy Check)}
Vengono visti i bit, \textit{D}, come un numero binario. Viene scelto un polinomio generatore di grado $r+1$ (una sequenza binaria in cui il primo elemento è sempre 1). L'obiettivo è trovare un numero \textit{r} di bits \textit{CRC} di ridondanza \textit{R}: viene calcolato così che $<D,R>$ (concatenazione) sia esattamente divisibile per \textit{G} (in modulo 2). Il ricevente conosce \textit{G}, e divide $<D^1,R^1>$ per \textit{G} e controlla che il resto di questa operazione sia proprio 0, il che gli garantisce che l'informazione non contenga errori. \\
Questo approccio è molto utilizzato in internet.