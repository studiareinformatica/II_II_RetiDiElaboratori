\section{Multimedia: audio}
I segnali audio analogici sono campionati a tasso costante (8000 campioni/secondo per il telefono, 44100 campioni/secondo per i CD, e così via \ldots); ogni campione viene quantizzato (arrotondato) ed ogni valore quantizzato viene rappresentato da un flusso di bit attraverso appositi \textit{codificatori vocali}.\\
Questi codificatori vocali possono essere di tre tipi:
\begin{itemize}
	\item waveform codecs
	\item source codecs (\textit{vocoders})
	\item hybrid codecs
\end{itemize}

\subsection{Waveform Codecs}\hfill\\
Non avendo nessuna conoscenza a priori di come il segnale sia stato generato, è necessario sapere la banda del segnale (minore ai 4KHz per la telefonia classica) e il massimo rumore di quantizzazione tollerabile.\\
Questo tipo di codifica è caratterizzato da un'alta qualità finale, una bassa complessità e un basso ritardo, e da robustezza agli errori e al rumore di fondo.

\paragraph{Pulse Code Modulation}
Questo metodo provvede a salvare i dati audio senza nessun tipo di compressione dati; la forma d'onda viene direttamente digitalizzata. In questo modo, i file risultanti sono di elevate dimensioni ma non richiedono elevata potenza di calcolo per essere riprodotti.\\
Si assume una banda  $ B = 4KHz $ e una frequenza di campionamento $ Bc=8KHz $m 8 bit/campione, 64kb/s.\\
L'onda è campionata a intervalli regolari sull'asse delle x; per ogni campione, uno dei valori disponibili sull'asse delle y è scelto da opportuni algoritmi (in genere viene approssimato al più vicino). Questo produce una rappresentazione discreta del segnale di ingresso che può essere facilmente codificata in digitale per essere memorizzata e manipolata. Queste operazioni sono effettuate da un unico circuito integrato chiamato \textit{ADC} (\textit{analog-to-digital converter}) a cui va fornito solamente il clock e l'alimentazione, in uscita produce direttamente la codifica digitale del segnale analogico in ingresso.\\ 
Per la \textit{demodulazione} (quindi per ricavare in ricezione il segnale campionato di ingresso), passato ogni periodo di campionamento, è letto il valore seguente e l'uscita si porta quasi istantaneamente al nuovo valore. Come risultato di queste variazioni istantanee si avrà un alto numero di armoniche indesiderate (maggiori di $ \frac{1}{2} f_c $ ) che si possono eliminare attraverso un filtro analogico. Teoricamente è impossibile rappresentare discretamente un segnale con banda infinita ma secondo il teorema del campionamento se si utilizza una frequenza di campionamento molto più alta di quella del segnale (pari a due volte la sua banda di frequenza) il segnale ricostruito in ricezione non si discosterà molto da quello originario. L'elettronica utilizzata per riprodurre un accurato segnale analogico da dati discreti è simile a quella usata per generare il segnale digitale. Questi dispositivi sono chiamati \textit{DAC} (\textit{digital-to-analog converters}), e operano in modo simile agli \textit{ADC}. Producono in uscita una tensione o una corrente (dipende dal tipo) secondo il valore presente in ingresso. Questo segnale di uscita viene generalmente filtrato e amplificato prima di essere utilizzato.\\
La quantizzazione del segnale può essere:
\begin{itemize}
	\item \textbf{Quantizzazione uniforme}	
	\begin{itemize}
		\item L'errore di quantizzazione è fisso (minore di $ \frac{q}{2} $, con $ q $ passo di quantizzazione)
		\item Servono 12 bit/campione per riuscire ad ottenere un errore di quantizzazione sufficiente; basso nel caso di valori piccoli.
	\end{itemize}
	\item \textbf{Quantizzazione non uniforme}
	\begin{itemize}
		\item Valori grandi possono sopportare errori maggiori
		\item Sono sufficienti 8 bit/campione in questo caso
	\end{itemize}
\end{itemize}

\paragraph{Differential PCM}
Aggiunge alcune funzionalità rispetto a PCM basandosi sulla predizione dei campioni del segnale. L'input può essere analogico o digitale.\\
Se l'input è un segnale analogico a tempo continuo, deve essere campionato prima per fare in modo che l'input dell'encoder DPCM sia un segnale a tempo discreto.\\
Abbiamo due alternative:
\begin{enumerate}
	\item prendere i valori di due campioni consecutivi; se sono campioni analogici, quantizzarli; restituire la differenza tra il primo e il secondo.
	\item invece di prendere la differenza relativa ad un campione precendente, prendere la differenza relativa all'output di un modello locale del processo di decodifica; in questa opzione la differenza può essere quantizzata - ciò consente di incorporare una perdita controllata nella codifica.
\end{enumerate}
Applicando uno di questi due processi, la ridondanza a breve termine del segnale è eliminata; i vantaggi di questo metodo sono una notevole riduzione della velocità di emissione (da 64Kbps a 32Kbps) e  una maggiore qualità \textit{datarate} disponibile per ogni canale vocale.\\
In questo modo, inoltre, i campioni vocali successivi sono correlati a quelli precedenti, ed è possibile utilizzare metodi di predizione per valutare il campione successivo noti i precedenti.\\
L'unica informazione trasmessa è la differenza tra valore predetto e valore reale. A causa della correlazione, la varianza della differenza è minore ed è possibile codificarla con un minor numero di bit.\\

\paragraph{Adaptive DPCM}
Questo tipo di compressione è un'estensione della codifica digitale PCM di base, usata largamente soprattutto nel campo della telefonia fissa.\\
A differenza della PCM di base, che campiona il segnale audio, lo quantizza secondo un codificatore non lineare e trasmette direttamente i valori quantizzati in formato numerico, l'ADPCM si basa sulla predicibilità di un campione a partire da un numero di campioni precedenti ad esso. È quindi sufficiente predire il campione n-esimo, valutare l'errore rispetto a quello reale e trasmettere soltanto l'errore di predizione. Dal canto suo, il decoder farà la stessa predizione e sommerà ad essa l'errore di predizione ricevuto. In questo modo il tutto funziona se la varianza dell'errore è minore della varianza dei campioni, nel senso che si avrà un reale risparmio di bit da trasmettere.\\
Con la tecnica ADPCM si riesce a garantire un bitrate di 32kbps in campo telefonico, contro i 64kbps del PCM tradizionale.

\subsection{Source Codecs}
Questi codificatori sono detti anche \textit{Vocoders} perchè si basano su modelli di generazione della voce umana che permettono di "togliere la ridondanza" da segmenti vocali fino ad un'informazione base sufficiente a riprodurre la voce.\\
Sono caratterizzati da un'elevata complessità, ritardi mediamente elevati e forte sensibilità ad errori, rumori di fondo e suoni non umani.

\section{Multimedia: video}
Un'immagine digitale è un array di pixel rappresentati da bit.\\
Un \textit{video} è una sequenza di immagini digitali visualizzate a tasso costante.\\
Per codificare un video si sfrutta la ridondanza spaziale (nelle immagini) e temporale (da un'immagine all'altra) per ridurre il numero di bit necessari per la codifica.

\paragraph{Streaming stored videos}
Nel progettare meccanismi di riproduzione di video locali, bisogna tenere conto di alcuni requisiti. Tra questi, uno dei più significativi ed impegnativi è quello del playout continuo: una volta iniziato il playout del client, il playback deve corrispondere al tempo originale, indipendentemente dai ritardi di rete (questo richiederà quindi un buffer client-side). Altri esempi sono l'interattività (possibilità di mettere in pausa, saltare da un fotogramma all'altro, mandare avanti velocemente) e la ritrasmissione di pacchetti persi.\\
% manca un pezzo

\section{Streaming multimedia: UDP}
Il server invia a tasso appropriato per il client. Spesso il tasso di invio e il tasso di codifica sono uguali e costanti; il tasso di trasmissione può essere indipendente dal livello di trasmissione. Il ritardo di playout è molto breve (solitamente da 2 a 5 secondi) per rimuovere il \textit{jitter}; per quanto riguarda l'aspetto dell'\textit{error recovery}, va ricordato che questo è effettuato solo a livello applicativo. Bisogna inoltre tenere conto del fatto che UDP potrebbe essere bloccato dai firewall.

\section{Streaming multimedia: HTTP}
Con lo streaming HTTP, i file multimediali sono recuperati direttamente tramite HTTP GETs e inviati via TCP a tasso più alto possibile.\\
È necessario un maggiore \textit{playout delay} che tenga conto della variabilità delle sorgenti nel tempo.\\
Un \textit{hyperlink} per un file multimediale contiene l'URL dell'audio/video in questione. Quando un utente clicca su di esso, la risposta include un header con il tipo di contenuto; il browser esamina l'header e lancia il riproduttore associato.\\
Con questo metodo la possibilità di perdita di informazioni è molto alta (\textit{playout buffer}, \textit{playout delay}, \textit{prefetching}, e così via \ldots). Per minimizzare gli errori di trasmissione sono state adottate delle tecniche di codifica che tentino il più possibile di limitare \textit{flipping}, adottando politiche che preferiscono molta ridondanza purchè ci sia \textit{error detection}. In questo caso, il \textit{throughput} varia da utente ad utente.

\paragraph{Dynamic Adaptive Streaming over HTTP}
Lo stesso contenuto è memorizzato sul server per codifiche diverse che variano in base all'endsystem che lo richiede. Solitamente è preferita una qualità più bassa per evitare errori di trasmissione.\\
Lato server, il file è diviso in \textit{chunks}, ognuno dei quali è conservato e codificato a diverse frequenze.\\
Il client misura la \textit{bandwidth}, consulta il manifest e richiede un chunk alla volta. Deve essere in grado di determinare quando richiedere il chunk, come richiederlo e \textit{a chi} richiederlo: i contenuti sono infatti replicati in batterie di server che possono essere geograficamente distribuiti su server propri o di hosting (CDN) secondo due logiche:
\begin{itemize}
    \item \textit{enter deep}: i server sono in prossimità degli ISP, vicino agli end users;
    \item \textit{bring home}: i server sono posizionati in cluster presso \textit{POP} (\textit{points of presence}).
\end{itemize}

\section{Voice Over IP}
Caratteristiche particolari:
\begin{itemize}
    \item forti requisiti in termini di ritardo (è considerato 'buono' fino a 150 millisecondi, 400 è il massimo tollerato)
    \item inizializzazione di sessione (determinare come raggiungere un utente)
    \item codifica
\end{itemize}
I pacchetti vengono generati solo se non c'è silenzio.\\
Si sceglie di trasmettere a 64Kbps solo se c'è segnale audio sul canale; altrimenti viene inviato un suono di sottofondo precedentemente campionato.\\
Tra i problemi di questa tecnica:
\begin{itemize}
    \item network loss
    \begin{itemize}
        \item router buffer overflow
        \item congestione
    \end{itemize}
    \item delay loss - datagrammi IP in ritardo
    \item tolleranza limitata per la perdita di pacchetti: da 1\% a 10\%
    \item i pacchetti che superano il ritardo tollerato vengono solitamente scartati
\end{itemize}
Il ricevente proverà a riprodurre ogni chunk esattamente \textit{q} millisecondi dopo la sua generazione (necessariamente $ q < 400msec $). Ogni chunk presenterà un timestamp t; il playout sarà $ p+q $. Appare quindi chiaro che più $ q $ è alto, minore è la perdita di pacchetti; più $ q $ è basso, migliore è l'esperienza utente (per le prestazioni).\\
Si parla di \textit{adaptive playout delay} perchè $ q $ varia in base al tipo di trasmissione; si avrà un ritardo maggiore appena finito un periodo di silenzio poichè impercettibile per l'utente.\\
Per quanto riguarda la perdita di informazioni, è adottata una \textit{Forward Error Correction}: sono inviati abbastanza bit da permettere il recupero delle informazioni dallo strato applicativo, senza bisogno di ritrasmissione.\\
Un esempio di \textit{FEC} può essere l'invio, ogni $ n $ chunk, di un chunk $ n + 1 $ di ridondanza calcolato come XOR dei precedenti.\\
Una \textit{FEC} molto utilizzata prevede che i pacchetti, normalmente inviati con codifica PCM a 64Kbps, presentino anche una codifica GSM a 13Kbps del pacchetto precedente. In questo modo, in caso di perdita di un pacchetto, si potrà comunque recuperare una sua versione a bassa qualità con il pacchetto successivo.\\
Uno degli applicativi VOIP più famosi, \textit{Skype}, utilizza UDP per la trasmissione dati, TCP per controllo e recovery (in caso il firewall blocchi UDP), la codifica FEC a livello applicativo e conserva inoltre le porte e gli IP (SN).

\paragraph{Real Time Streaming Protocol}
Questo protocollo risiede sugli endsystem (tra UDP e lo strato applicativo) e specifica la struttura dei pacchetti trasportanti audio/video. La sua unica funzione è estendere l'header dei pacchetti per fornire:
\begin{itemize}
    \item time stamping
    \item sequence numbering
    \item identificazione tipo di payload
    \item IP/porte
\end{itemize}

\paragraph{Real Time Control Protocol}
Com'è facilmente intuibile, solitamente è associato a RTSP e funge da canale di controllo dedicato.\\
Ogni fruitore di RTSP invia periodicamente pacchetti di controllo RTCP contenenti report/feedback (numero di pacchetti inviati e ricevuti, \textit{jitter}, e così via \ldots). Tenta di non consumare più del 5\% della banda totale.
