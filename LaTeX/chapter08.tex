Per anni - nonostante già esistessero le reti mobili - si accedeva ad internet con dispositivi fissi cablati. Negli ultimi anni, invece, la situazione si è evoluta drasticamente, grazie all'arrivo degli smartphones: si è passati da un accesso cablato ad un accesso wireless/mobile ed è cambiato il tipo e l'intensità di traffico, molto più multimediale.
Diventa quindi di rilevante importanza la questione della gestione dei consumi energetici. \\
I dispositivi che accedono ad internet sono molto variegati: da desktop/laptop, a palmari, a smartphone, a dispositivi dell'\textit{internet delle cose}. Tutto risulta essere connesso ed interagire con internet. I punti di accesso ad internet diventano sempre più radio. \\
Occorre quindi imparare a gestire questa tipologia di accesso, che si articola di due tipi:
\begin{itemize}
	\item wireless \textit{anywhere/anytime}: possibilità di accedere ad internet da mobile anche se in movimento;
	\item wireless senza mobilità (rete universitaria): un link wireless (un \textit{access point}, o \textit{relay})che offre accesso wifi, collegato a sua volta - attraverso una parabola ad alta ricezione (come per esempio la tecnologia \textit{WiMAX})- ad un dispositivo radio centrale (la \textit{stazione base}).
\end{itemize}

\paragraph{Nomadic computing}
Insieme delle circostanze riguardanti un endpoint che si connette - anche con accesso cablato - in reti differenti (eventualmente con IP diversi).

\paragraph{Mobile computing}
Insieme delle circostanze riguardanti endpoint non connessi in maniera cablata. \\

Il link che connette i dispositivi alla \textit{stazione base} lavora a livello due della pila \textit{TCP/IP}, è utiizzato come \textit{backbone link} e garantisce accesso multiplo attraverso protocolli che ne coordinano l'accesso. \\
Esistono moltissime alternative per le tecnologie di link wireless, che garantiscono copertura di aree più o meno vaste:
\begin{itemize}
	\item \textit{802.11n}: 10m - 30m;
	\item \textit{802.11a,g}: 10m - 30m;
	\item \textit{802.11a,g point-to-point}: 50m - 5km;
	\item \textit{802.11b}: 10m - 30m;
	\item \textit{802.15}: 10m - 25m;
	\item \textit{802.16} (\textit{WiMAX}): 30m - 20km;
	\item \textit{UMTS/WCDMA-HSPDA}, \textit{CDMA2000-1xEVDO}: 30m - 15km;
	\item \textit{UMTS/WCDMA}, \textit{CDMA2000}: 50m - 15km;
	\item \textit{IS95}, \textit{CDMA}, \textit{GSM}: 50m - 15km.
\end{itemize}

\paragraph{Reti infrastrutturali}
Nelle reti infrastrutturali vi sono terminali mobili, un'infrastruttura e dei punti di accesso ad essa, garantendo tipicamente accesso ad internet.
\begin{itemize}
	\item Profondità a singolo hop: l'host si connette ad una \textit{stazione base} che si connette ad internet;
	\item Profondità ad hop multipli: l'host può fare da \textit{relay} per altri nodi per permettergli di connettersi a loro volta ad internet.
\end{itemize}

\paragraph{Reti \textit{ad hoc} (non infrastrutturali)}
Nelle reti \textit{ad hoc} non esiste \textit{stazione base} e i nodi possono soltanto trasmettere agli altri nodi nell'arco di coperatura del link wireless a cui sono connessi. Inoltre i nodi si organizzano in una network indipendente, in cui essere capaci di stabilire delle rotte funzionanti.
\begin{itemize}
	\item Profondità a singolo hop: non c'è né \textit{stazione base}, né connessione ad internet;
	\item Profondità ad hop multipli: non c'è né \textit{stazione base}, né connessione ad internet, ma alcuni nodi potrebbero ugualmente funzionare da \textit{relay} per permettere accessi da alcuni nodi, ad altri. Si tratta di reti \textit{manet} e/o \textit{vanet}.
\end{itemize}

\section{Caratteristiche dei link Wireless}
Le differenze sostanziali dai link cablati:
\begin{itemize}
	\item La densità di potenza del segnale trasmesso - propagandosi - decresce;
	\item Può verificarsi interferenza causate da sorgenti non necessariamente coinvolte direttamente nella connessione (e.g., dispositivi a micro-onde, motori, \ldots);
	\item Propagazione \textit{multipath}: il segnale radio in base alla diffrazione che ne varia il percorso, arriva in momenti diversi in punti diversi.
\end{itemize}
Tutte queste differenze rendono la comunicazione wireless molto più difficile da gestire rispetto alla comune rete cablata.

\paragraph{SNR}
Il \textit{signal-to-noise ratio} è il rapporto signale-rumore, che deve essere entro una certa soglia per poter garantire ricezione. Chiaramente può variare nel tempo. E' importante perché è collegato alla probabilità di errore. \\

Il link wireless rappresenta un canale \textit{broadcast}: a creare nuovi problemi è la gestione della interazioni multiple tra un \textit{sender} ed un \textit{receiver} wireless. Il motivo per cui non è possibile utilizzare il protocollo \textit{CSMA/CD} è rappresentato dal problema dell'\textit{hidden terminal}: vi sono tre terminali \textit{A}, \textit{B} e \textit{C}, dove \textit{A} e \textit{B} possono comunicare senza problemi, così come \textit{B}, \textit{C}, ma non come \textit{A} e \textit{C}, perché esiste un impedimento fisico tra i due. Ancora può verificarsi il problema dell'attenuazione del segnale, dove \textit{B} si trova nel raggio trasmissivo di \textit{A} e \textit{C}, ma \textit{A} e \textit{C} non sono nel raggio trasmissivo l'uno dell'altro; anche se decidessero comunque di provare a trasmettersi, porterebbero ad una collisione su \textit{B}. \\
Un'alternativa sviluppata è il Code Division Multiple Access (CDMA): viene utilizzato in diversi standard per canali di broadcast wireless. Ad ogni utente viene assegnato un codice univoco, tutti gli utenti condividono la stessa banda di frequenza e ad ogni utente viene assegnata una sequenza che lo autentichi (come il codice univoco): quando avviene la trasmissione, viene moltiplicato quanto da trasmettere con la sequenza, detta \textit{chipping code}. CDMA passa per tutta una serie di difficoltà, che permettono il sovrapporsi di trasmissioni, garantendo la capacità di decomporle autonomamente.

\paragraph{IEEE 802.11}
Diverse tecnologie (più è alto lo spettro della banda, più si è aperti ad attenuazioni), che però usano tutte il protocollo \textit{CSMA/CA} per l'accesso multiplo e tutte supportano sia una rete di tipo infrastrutturale che non:
\begin{itemize}
	\item 802.11b: spettro non licenziato tipicamente di 2.4-5GHz, raggiunge un massimo di 11Mbps e tutti gli host usano lo stesso chipping code;
	\item 802.11a: spettro di 5-6GHz, con un tetto di 54Mbps;
	\item 802.11g: spettro di 2.4-5GHz, con un tetto di 54Mbps;
	\item 802.11n (con più trasmettenti): spettro di 2.4-5GHz, con un tetto di 200Mbps.
\end{itemize}

In una LAN wireless, il \textit{BSS} (\textit{Basic Service Set}) è una cella dell'infrastruttura, che contiene gli host wireless, l'access point (AP) e che supporta la modalità \textit{ad hoc}. \\
Un'espansione del \textit{CSMA/CA} prevede che un comportamento diverso da parte del sender: se il canale è libero per il \textit{DIFS} (ovvero il \textit{Distributed InterFrame Spacing}), allora trasmetto l'intero frame, altrimenti aspetto per un intervallo casuale di temp, al termine del quale trasmetto soltanto se il canale è libero, altrimenti lo raddoppio. Una volta trasmesso il dato, il ricevitore risponde con un ACK dopo un \textit{SIFS} (sempre minore del \textit{DIFS}, per garantire che altri non confondano questo tempo come canale libero). \\
Per evitare le collisioni invece, il sender prima trasmette un piccolo pacchetto chiamato \textit{RTS} (\textit{request-to-send}): se questo non cattura collisioni, allora apro la trasmissione effettiva. In risposta all'\textit{RTS}, vi è il \textit{CTS} (\textit{clear-to-send}?).
Poiché questo flusso porta ad un \textit{overhead} dovuto alle trasmissioni di \textit{RTS} e \textit{CTS}, \textit{802.11} include una ottimizzazione sull'adattabilità alla banda.

\paragraph{Trame}
Il formato della trama in \textit{802.11} è dato da:
\begin{itemize}
	\item 2 byte di \textit{frame control};
	\item 2 byte di \textit{duration};
	\item 6 byte per ciascun indirizzo MAC:
	\begin{itemize}
		\item \textit{indirizzo 1}: destinazione (indirizzo MAC del wireless host o AP che riceve il frame);
		\item \textit{indirizzo 2}: sorgente (indirizzo MAC del wireless host o AP che trasmette il frame);
		\item \textit{indirizzo 3}: indirizzo MAC dell'interfaccia router al quale l'AP è conneso (colui che fa da ponte per internet, per esempio).
	\end{itemize}
	\item 2 byte di \textit{seq control};
	\item 6 byte per l'\textit{indirizzo 4} (usato solo nella modalità ad hoc);
	\item 0-2312 byte di \textit{payload};
	\item 4 byte di \textit{CRC}.
\end{itemize}
Il dialogo tra l'access point e l'interfaccia router avviene secondo \textit{802.3}, il cui formato della trama prevede vi siano solo due indirizzi: l'indirizzo MAC di destinazione (MAC del router) e quello mittente (MAC dell'access point).

\paragraph{Gestione energetica}
Per migliorare i consumi energetici, i nodi allertano l'AP che \textit{vanno a dormire} fino al successivo \textit{beacon frame}, così che l'AP sa di non dover trasmettere frame a quel nodo e il nodo stesso non \textit{si sveglia} entro il successivo \textit{beacon frame}.
Il \textit{beacon frame} (o \textit{pacchetto faro}) è definito come il frame che contiene una lista di device con frame \textit{AP-to-mobile} che attendono l'invio: ciascun nodo rimarrà \textit{sveglio} se i frame sono inviati; altrimenti in \textit{sleep} fino al prossimo \textit{beacon frame}. \\
Il \textit{duty cycle} è definito come l'intervallo che intercorre tra il tempo di \textit{ON} e il tempo di \textit{ON} sommato al tempo di \textit{OFF}.

\section{Tecnologia Bluetooth}
Lo standard associato è l'\textit{802.15} e il suo obiettivo è quello di avere un raggio di azione più corto rispetto all'\textit{802.11}.
La modalità di comunicazione è \textit{master/slave}, dove gli slave fanno richieste unicamente al master, che risponde a ciascuna di esse.
Si tratta di una rete \textit{ad hoc}, in cui la trasmissione ha una banda pari a 2,4-5 GHz e raggiunge una capacità trasmissiva massima di 721kbps. \\
Il primo standard fu rilasciato nel 1999. Il \textit{Bluetooth Special Interest Group} conta oltre 1800 membri, tra cui Ericsson, Nokia, IBM, Intel, Toschiba, Microsoft, Lucent, 3Com, Motorola, e così via \ldots \\
La cosa più interessante è che viene adoperato un approccio totalmente nuovo: il \textit{Frequency Hopping Spread Spectrum}. I deviecs seguono una sequenza \textit{FHSS} pseudocasuale generata utilizzando l'ID del master (accessibile non solo al msater ma anche a tutti gli slaves). La frequenza usata per la trasmissione cambia per ogni pacchetto, garantendo scarsa interferenza ed elevata sicurezza. \\
I devices bluetooth sono organizzati in \textit{piconets}, cluster formati da un master e diversi slaves (per un massimo di 7 in comunicazione). La formazione di una \textit{scatternet} (interconnessione di \textit{piconet}) è definita da tre problemi fondamentali:
\begin{itemize}
	\item la scoperta di devices (tramite la procedura bluetooth standard \textit{inquiry}/\textit{inquiry scan});
	\item la formazione di una \textit{piconet};
	\item l'interconessione delle \textit{piconet}.
\end{itemize}
Di fondamentale importanza è il fatto che master sia appartenenete in una \textit{piconet} alla volta, perché altrimenti l'attività presso di una comprometterebbe la necessità dell'attività dello stesso master in un'altra \textit{piconet}. \\
Il punto dolente della tecnologia bluetooth è la \textit{device discovery}: impiega troppo tempo per il detecting ed è troppo altalenante nel mantenere la connnessione. Questa fase è definita dall'\textit{inquiry scan}: viene fatto uno scan per individuare tutti i possibili nodi, ai quali viene mandato un pacchetto \textit{FHS} che mette al corrente dell'ID del master, così da permettere il calcolo della sequenza. Un nodo può essere o in modalità \textit{inquiry} o \textit{inquiry scan}: si può stabilire quindi una connesione soltanto tra nodi in modalità opposta. Per questo ogni nodo salta costantemente da uno stato di \textit{inquiry} a \textit{inquiry scan}, e viceversa. \\
Se esistono piu di 7 nodi in una \textit{piconet}, occorre formare una \textit{scatternet}. Per gestirle coerentemente, vengono utilizzati due tipi di scheduling: lo scheduler \textit{interpiconet} e \textit{intrapiconet}. Il primo si occupa di gestire quando e per quanto tempo considerare uno slave come attivo in una data \textit{piconet}. Il secondo invece determina lo schema da adottare in ciascuna \textit{piconet}: per esempio, in maniera proporzionale alla quantità di dati trasmessi da un device, saranno allocate risorse per questo, in modo tale da evitare di fare \textit{polling} continuo su device poco attivi, risparmiando risorse. L'obiettivo di entrambi è fare \textit{load balancing}.

\section{Rete cellulare}
La rete cellulare è garantita tramite il tassellamento di un'area con \textit{stazioni base} in modo tale da garantire copertura per la totalità del territorio. \\ Gruppi di stazioni base, le celle, sono collegate ad un \textit{mobile switching center} (o \textit{MSC}), al quale trasmettono tramite \textit{FDMA} e \textit{TDMA} (lo spettro viene diviso in canali di frequenza, a loro volta divisi in slot di tempo), nel caso di infrastrutture precedenti alla terza generazione (nelle successive viene usato il \textit{CDMA}), e che a sua volta è collegato alla rete telefonica e/o internet pubblica del gestore. \\
Nel passaggio delle generazioni sono migliorate le velocità di tramissione e qualità di trasmissione della voce e dei dati. Infatti, nella generazione \textit{2G} (in cui era possible trasmettere soltanto la voce), tra le celle e l'\textit{MSC} era frapposta una \textit{BSC} (\textit{Base Station Controller}), che gestiva soltanto un numero molto limitato di celle. \\
Nelle infrastrutture delle generazioni successive, dal \textit{2.5G} in poi (in cui fu introdotta la possibilità di trasmissione di dati, oltre alla voce), ogni \textit{BSC} è collegato non solo ad un \textit{MSC} per il supporto voce, ma anche ad un \textit{SGSN} (\textit{Serving GPRS Support Node}) per il supporto internet, che è collegato a sua volta ad un \textit{GGSN} (\textit{Gateway GPRS Support Node}) che fa effettivamente da ponte verso internet. \\
Nelle generazioni successive al \textit{2.5G}, è stato modificata la struttura per raggiungere miglioramenti nella qualità e nell'intensità del servizio offerto. \\ \\

Uno dei problemi più importanti è quello della mobilità, in quanto occorre garantire una connessione continua anche a device in costante movimento, che si agganciano/sganciano continuamente a/da celle distinte adiacenti. Come garantire che, nonostante tutti questi spostamenti, un device qualsiasi sia comunque capace di definire un indirizzo attraverso cui raggiungere il device in costante movimento? Vi sono due approcci:
\begin{itemize}
	\item \textit{let routing handle it}: il router richiede l'aggiornamento delle \textit{routing table}, così da mantenerle costantemente aggiornate sugli spostamenti dei devices. E' una soluzione decisamente non scalabile su milioni di devices;
	\item \textit{let end-system handle it}: lo stesso end-system notifica ogni spostamento ai vari router.
\end{itemize}
Sostanzialmente, ogni device ha una \textit{home network} (la propria rete, la rete casalinga), dentro la quale gli viene assegnato un IP permanente che è sempre utilizzato per raggiungere il device. All'interno della stessa rete c'è un \textit{home agent}, il responsabile della rete. Quando ci si muove, si raggiunge una \textit{visited network}, dove il responsabile è un \textit{foreign agent}: al device viene assegnato un IP temporaneo \textit{care-of-address} valido in quella \textit{visited network}. Come può un qualsiasi altro device a raggiungere quel device? Nel momento in cui quest'ultimo entra in una \textit{visited network}, deve contattare il \textit{foreign agent} per avvertirlo che si trova ora in quella rete e che ha un dato indirizzo permanente. L'agent utilizzerà quell'indirizzo per raggiungere il suo \textit{home agent}, segnalandogli lo spostamento. Allora il device esterno che desiderava contattarlo, conoscendo l'IP permanente del device, sarà in grado di contattare la \textit{home network} che lo reindizzerà alla \textit{visited network}. \\
Questo sistema di routing è detto \textit{routing indiretto}: l'inefficienza è data dal fatto che se due device si trovano entrambi in una stessa \textit{visisted network}, per contattarsi passeranno comunque ad interrogare la \textit{home network} di ciascuno. L'alternativa è il \textit{routing diretto}: viene fatta - in fase di creazione di una connessione tra due devices - richiesta all'\textit{home agent} di ciascuno. Una volta ottenuta la risposta, non viene più richiesto, e viene utilizzato quello che ho chiesto in fase di configurazione. Il rischio è che, se nel frattempo uno dei devices si sposta, l'indirizzo risulterà sbagliato. Allora si introduce il concetto di \textit{anchor foreign agent}, il precedente \textit{foreign agent} di un device, il quale verrò contattato in ogni spostamento dello stesso device in nuove \textit{visited network}.

\paragraph{Agent discovery}
La scoperta dell'agent della rete avviene attraverso il broadcasting di un messaggio \textit{ICMP}, detto \textit{advertisement}. La notifica degli agent invece per l'avviso del nuovo indirizzo nella \textit{visited network} invece è fatto tramite datagrammi \textit{UDP} sulla porta 434.