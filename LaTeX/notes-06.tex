\section{Connection, connection-less services}
Datagram: la rete fornisce un servizio a livello rete \textit{connectionless}
Virtual-Circuit: in una fase di setup della connessione viene determinato il percorso da seguire e mantiene delle informazioni a livello di ogni elemento di commutazione del router.

VC:
	percorso
	identificativo (vc number/s), uno per path
%	un'altra cosa che non ho fatto in tempo a scrivere.

Nella fase di setup, all'arrivo di pacchetto, la tabella di instradamento dice dove reinstradarlo e cambia id del vc. 
\textbf{vc routers maintain connection state informations.}
signaling protocols.
used to setup, maintain, teardown vc. not used in today's internet.

datagram networks:
	no etup.
	router non hanno stato su connessione end-to-end; in pratica non c'è nessun concetto di "connessione" a livello rete. I pacchetti vengono inoltrati %...


LONGEST PREFIX MATCHING
when looking for forwarding table entry for given dest address, use \textit{longest} address preix that matches dest address.


Router.
due funzioni chiave:
	far girare algoritmi/protocolli\footnote{RIP,OSPF, BGP} di routing;
	inoltrare datagrams da collegamneti incoming a outgoing+

	router: Input ports
		decentralized switching
		tre parti: 	
					line termination
					link layer protocol (receive)
					lookup, forwarding, queueing	mi arriva un datagramma da instradare, devo avere spazio
													in buffer da poter gestire
		
		Per un discorso di efficienza una copia della fwdt verrà memorizzata temporaneamente

		Switching fabrics
		trasferiscono pacchetti da input buffer a out buff appropriato
		switching rate: tasso al quale i pacchetti possono essere trasferiti; spesso misurato come multiplo di line rate
		tre tipi di sf:
			|memory|bus|crossbar|
			
			memory:
				router di prima generazione. pacchetti copiati su memoria di sistema; velocità limitata da bandwith di memoria
			bus:
				n linee di ingresso, n uscita, bus condiviso
				posso inoltrare verso l'output port un unico datagramma  alla volta!
				perdite di pacchetto?!
			crossbar (rete di interconnessione):
				n linee ingresso, n uscita, 2n segmenti con tutte possibilità di interconnessione tra in e out
				parallelismo; multipli datagrammi alla volta.
				
	router: Output ports
		datagram buffer, queueing
		link layer protocol, send
		line termination
		
		buffering required when datagrams arrive from fabric faster than transmission rate